Для автоматизации сборки программ существуют две полезные вещи --- утилита make и Makefile.

Утилита make предназначенна для автоматизации преобразования файлов из одной формы в другую. Правила преобразования задаются в скрипте с именем Makefile, который должен находиться в корне рабочей директории проекта).

Рассмотрим структуру Makefile на примере.

\begin{shCode}{Makefile}
	CC=gcc
	CFLAGS=-m64

	all: $(PROJS)
		@echo Done!
	main:
		$(CC) $(CFLAGS) -o $@ $(@:=.c) \end{shCode}

В начале объявляются переменные, затем конструкция вида:

<цель> : <зависимости> \\
<список команд>
			
\begin{shCode}{Утилита make}
	ag@helios:/home/ag $ make
	gcc -m64 -o main main.c
	Done!
	ag@helios:/home/ag $ ./main
	Inspiration unlocks the future \end{shCode}

При команде \textbf{make <цель>} будут выполнены команды соответствующие метке <цель>, а также последовательно все зависимости(другие метки), если они не являются файлами.

В нашем примере make по умоляанию перейдет к цели all, увидит зависимость main и перейдет на эту метку. Далее так как зависимостей у main нет, make выполнит команду “gcc -m64 -o main main.c“, а затем вернется к метке all и выполнит команду echo.

Makefile имеет специальные собственные переменные.

Приведем некоторые из них:

\begin{itemize}
	\item \$@  --- имя текущей цели;

	\item \$<  --- имя первой зависимости Makefile;

	\item \$\textasciicircum  --- имена всех зависимостей.
\end{itemize}