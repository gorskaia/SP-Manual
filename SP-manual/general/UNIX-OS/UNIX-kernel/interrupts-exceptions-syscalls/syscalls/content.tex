\begin{defi}{Системный вызов}
	 это запрос пользовательского процесса на выполнение какой-либо функции ядра от имени этого процесса.
\end{defi}

Процессы не могут напрямую обращаться к ядру и должны использовать интерфейс системных вызовов. После того как процесс производит системный вызов, запускается специальная последовательность команд (называемая переключателем режимов), переводящая систему в режим ядра, а управление передается ядру, которое обрабатывает операцию от имени процесса.

\textbf{В чем разница между прерываниями и системными вызовами?}

При выполнении программного прерывания (int) у вас сразу же происходит переход к обработчику этого прерывания по вектору, который указан в инструкции. Код всех обработчиков лежит в оперативной памяти по заранее известным адресам, которые были выбраны разработчиками архитектуры. Внутри этих обработчиков и происходит обработка прерываний. 

Надо понимать, что когда вы вызываете прерывание, происходит очень много аппаратной работы по сохранению контекста. Это довольно дорогая операция, поэтому со временем решили отказаться от использования “int 0x80” и использования механизма прерываний для системных вызовов в пользу новой инструкции syscall. 

Инструкция syscall позволяет перейти к обработчику, который расположен в пользовательском пространстве, и попытаться обработать системный вызов, который вы хотите совершить. Только потом, если что-то понадобится от системы, можно обратиться к ней непосредственно.