Немного о том, как программы отличают файлы, с которыми они взаимодействуют. Для обращения программы к файлу в ОС UNIX существует специальная структура --- файловый дескриптор. 

\begin{defi}{Файловый дескриптор}
	структура описывающая файл, с которым взаимодействует программа.
\end{defi}

Каждый раз, когда программа (процесс) создает новый поток ввода-вывода (например, при открытии файла), ядро создает новый файловый дескриптор в таблице открытых файловых дескрипторов. Таблица файловых дескрипторов существует для каждого отдельного процесса и находится в его u\_block области, то есть один и тот же дескриптор у разных процессов может ссылаться на разные файлы.

\begin{center}
	\begin{tabular}{c|c|c}
		\textbf{Номер дескриптора} & \textbf{Имя файла} & \textbf{Флаги доступа} \\
		\hline
		... & ... & ... \\
		\hline
		42 & /etc/passwd & O\_RDWR \\
		\hline
		43 & /home/ag/secrets & O\_RDONLY \\
	\end{tabular}

	\caption{Структура таблицы открытых файловых дескрипторов}
\end{center}

Как можно видеть, каждая запись о файловом дескрипторе в таблице файловых дескрипторов имеет:

\begin{itemize}
	\item номер --- целое неотрицательное число не более константы OPEN\_MAX, определяемой ОС;
	\item символьное имя (путь) файла в файловой системе;
	\item флаги доступа к файлу (каким образом данный процесс может с этим файлом взаимодействовать).
\end{itemize}

\begin{important}
	Также в таблице файловых дескрипторов процесса по умолчанию уже открыты стандартные потоки ввода, вывода и ошибок. Они имеют номера файловых дескрипторов 0, 1 и 2 соответственно (в стандарте POSIX.1 используются константы STDIN\_FILENO, STDOUT\_FILENO и STDERR\_FILENO).

	Вы можете точно также закрыть, открыть только на чтение и т.д. любой из стандартных потоков.
\end{important}

После создания файлового дескриптора ОС возвращает процессу номер этого дескриптора в таблице открытых файловых дескрипторов для данного процесса. В дальнейшем, процесс может обращаться к файлу по номеру возвращенного файлового дескриптора, тем самым абстрагируясь от файлов, с которыми он взаимодействует.

\textit{Иногда для упрощения речи говорят: открой 8-ой дескриптор для записи файла. Это не совсем верно. Правильнее было сказать открой дескриптор с номером 8 на запись в файл, так как файловый дескриптор это все-таки структура, а не число.}