Внутри директории точно так же есть смещение, с которым не очень удобно работать с с помощью системного вызова. Его нужно будет декрементировать и инкрементировать. Для упрощения этой задачи, существует набор функций, реализованных в libc. 

Функция opendir (3) открывает директорию для чтения с именем name и возвращает указатель на directory stream. При ошибке возвращает NULL. Когда мы вызываем opendir мы получаем не привычный нам dirhandler. Собственно, он является указателем на директорию, с которой мы собрались работать. Аргумент dirname -- это путь к директории, которую мы хотим открыть.

\begin{CCode}{opendir (3)}
	#include <dirent.h>

	DIR * opendir(
		const char *name
	); \end{CCode}

Для последующего чтения директорий есть readdir (3). Эта функция возвращает следующую структуру dirent считанную из файла-директории. При достижении конца списка файлов в директории или возникновении ошибки возвращает NULL

\begin{CCode}{readdir (3)}
	#include <dirent.h>

	struct dirent * readdir(
		DIR *dirp
	); \end{CCode}

После того как мы поработали с директорией мы можем ее закрыть для этого у нас есть closedir (3). Эта функция закрывает поток директории.

\begin{CCode}{opendir (3)}
	#include <dirent.h>

	int closedir(
		DIR *dirp
	); \end{CCode}