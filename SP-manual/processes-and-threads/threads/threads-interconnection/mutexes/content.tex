\begin{defi}{Мьютекс}
	простейший объект синхронизации, имеющий два состояния: «заблокирован» и «свободен».
\end{defi}

При использовании мьютекса только один поток в определенный момент времени может заблокировать мьютекс и получить доступ к разделяемому ресурсу. При завершении работы с ресурсом поток должен разблокировав мьютекс. 

Реализованы pthread\_mutex. 

\begin{CCode}{Приведем некоторые функции для работы с мьютексами:}
	#include <pthread.h>

	int pthread_mutex_init(pthread_mutex_t *mutex, 
            const pthread_mutexattr_t *attr);
	
	int pthread_mutex_destroy(pthread_mutex_t *mutex);

	int pthread_mutex_lock(pthread_mutex_t *mutex);

	int pthread_mutex_trylock(pthread_mutex_t *mutex);

	int pthread_mutex_unlock(pthread_mutex_t *mutex); \end{CCode}

Вам предлагается ознакомиться с ними самостоятельно.

Надо понимать, что используя мьютексы, вы можете эмулировать rwlock . Используя rwlock, вы можете эмулировать семафоры. Используя семафоры --- любое поведение. Самое главное --- вам нужно определиться, что и для какого случая использовать. Они условно взаимозаменяемы, но всё зависит от того, что конкретно вам нужно от кода.