\begin{defi}{Очередь сообщений}
	механизм передачи некоторого объема данных между процессами, основанный на сигналах
\end{defi}

Очередь сообщений используется, когда вам нужно передать какому-то клиенту или блоку информацию о том, что необходимо обработать. Позволяет передавать небольшие объемы данных между процессами. 

Очередь могут использовать все процессы, имеющие ключ и обладающие правами доступа к очереди (msg\_perm).

Структура сообщения очереди имеет следующие поля:

\begin{itemize}
	\item тип;
	\item длина сообщения в байтах;
	\item тело сообщения.
\end{itemize}

\begin{CCode}{Для реализации очередей используются такие системные вызовы как:}
	#include <sys/msg.h>

	int msgctl(int	msqid, int cmd,	struct msqid_ds	*buf);
	int msgget(key_t key, int msgflg);
	int msgsnd(int	msqid, const void *msgp, size_t	msgsz, int msgflg); \end{CCode}