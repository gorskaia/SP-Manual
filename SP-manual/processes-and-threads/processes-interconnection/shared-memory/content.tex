\begin{defi}{Разделяемая память}
	механизм для работы с общей для нескольких процессов памятью. Сегмент разделяемой памяти хранится до его принудительного удаления.
\end{defi}

Один из способов присоединения участка физической памяти к виртуальному адресному пространству различных процессов.
	
Сущестувует два способа реализации межпроцессного взаимодействия через разделяемую память --- System V и POSIX. В обоих случаях в качестве итога вы получаете некоторый указатель на логическое адресное пространство процессов, в рамках которого вы можете работать с общим сегментом.

Работа с разделяемой памятью в POSIX осуществляется с помощью системных вызовов mmap (2) и munmap (2). Они уже упоминались в части “Память“.

В отличие от POSIX реализации, в которой разделяемую память нужно отображать в файл, System V осуществляет доступ к разделяемой памяти по специально сгенерированному или установленному ключу.

В System V механизм разделяемой памяти реалзован системными вызовами shmget (2), shmat (2), shmdt (2).

Системный вызов shmget (2) для создания разделяемой памяти или получению доступа к уже существующей. Этот системный вызов принимает ключ -- идетификатор разделяемой памяти, размер в байтах и флаги режима доступа.

\begin{CCode}{shmget (2)}
	#include <sys/shm.h>

	int shmget(key_t key, size_t size, int flag); \end{CCode}

Системный вызов shmat (2) принимает полученный вызовом shmget (2) файловый дескриптор, желаемый адрес отображения, и флаги (например, SHM\_RDONLY). Если адрес установлен в 0, то система сама выберет адрес отображения. 

\begin{CCode}{shmat (2)}
	#include <sys/ipc.h>
	#include <sys/shm.h>

	void * shmat(int shmid, const void *addr, int flag); \end{CCode}

Системный вызов shmdt (2) используется для отключения области разделяемой памяти. В качестве единственного аргумента он принимает указатель адрес отображения.

\begin{CCode}{shmdt (2)}
	#include <sys/ipc.h>
	#include <sys/shm.h>

	int shmdt(const void *addr); \end{CCode}