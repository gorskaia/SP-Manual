Когда мы говорим о ссылках, чаще всего мы говорим о жестких ссылках. Важно понимать, что внутри каталога могут быть две и более жестких ссылок, ссылающихся на один и тот же inode. Ссылка --- это логическое понятие, inode - физическое. Таким образом, мы можем говорить, что файл доступен по двум путям.

Еще раз. Имя не является атрибутом файла, оно является алиасом, по которому мы можем получить номер индексного дескриптора. (Читайте про структуру dirent)

Системный вызов link (2) создает жесткую ссылку на файл. В качестве аргументов link (2) принимает путь к файлу, на которой мы хотим создать ссылку, и, соответственно, путь для этой самой ссылки.

\begin{CCode}{link(2)}
	int link( 
		const char *existing,	/* path to file */ 
		const char *new			/* path to link */ 
	); \end{CCode}
Возвращает: 0 или код ошибки.

\begin{important}
	Жесткую ссылку возможно создать только в пределах одной файловой системы.
\end{important}

Системный вызов unlink (2) позволяет удалить жесткую ссылку на файл. В качестве аргумента unlink (2) принимает путь к файлу, по которому мы хотим удалить ссылку. Возвращает 0 или код ошибки.

\begin{CCode}{unlink (2)}
	int unlink(
		const char *path
	);	\end{CCode}

Создание и удаление жесткой ссылки состоит из двух операций:

\begin{itemize}
	\item {модификация каталога. Для этого у нас должны быть права на запись в каталог;}
	
	\item {модификация inode, который описывает этот файл --- мы должны увеличить на 1 nlink при создании жесткой ссылки и уменьшить на 1 nlink при удалении.}
\end{itemize}

Вопрос: что сделать первым? Сначала мы должны изменить директорию. Чтобы у нас не получилось ситуации, когда жесткая ссылка в каталоге ссылается на файл, которого нет.

\begin{important}
	Файл будет являться удаленным, когда на него не останется ни одной жесткой ссылки (nlink = 0).
\end{important}