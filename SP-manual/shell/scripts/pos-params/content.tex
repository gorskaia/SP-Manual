\begin{defi}{Позиционные параметры}
	это аргументы, передаваемые скрипту при запуске из терминала, и имя самого скрипта. 
\end{defi}

\begin{shCode}{Например}
ag@helios:/home/ag$ ./myscript 16 name 4 \end{shCode}
	myscript - имя скрипта, 16, name, 4 - аргументы, переданные скрипту.

Внутри самого скрипта можно посмотреть с помощью позиционных переменных: \textbf{\$0, \$1, \$2, \$3,...}, где
	\$0 -- имя файла запущенного скрипта} \\
	\$1, \$2, \$3..	 --- аргументы, передаваемые скрипту.

\begin{important}
	аргументы, следующие за \$9, должны заключаться в фигурные скобки, например: \$\{10\}, \$\{11\}, \$\{12\}. Отсутствие фигурных скобок может привести к интерпретированию только первой цифры в качестве позиционного параметра (Например \$89 как \$8 и цифра 9).
\end{important}
