\documentclass[10pt,pdf,hyperref={unicode}]{beamer}

\usepackage[utf8]{inputenc}
\usepackage[english,russian]{babel}

\usepackage{amsmath,amsthm,amssymb}
\usepackage{mathtext}
\usepackage[T1,T2A]{fontenc}

% special
\usepackage{xcolor}
\usepackage{listings}

\definecolor{mGreen}{rgb}{0,0.6,0}
\definecolor{mGray}{rgb}{0.5,0.5,0.5}
\definecolor{mPurple}{rgb}{0.58,0,0.82}
\definecolor{backgroundColour}{rgb}{0.95,0.95,0.92}

\lstdefinestyle{CStyle}{
    backgroundcolor=\color{backgroundColour},   
    commentstyle=\color{mGreen},
    keywordstyle=\color{magenta},
    numberstyle=\tiny\color{mGray},
    stringstyle=\color{mPurple},
    basicstyle=\footnotesize,
    breakatwhitespace=false,         
    breaklines=true,                 
    captionpos=b,                    
    keepspaces=true,                 
    numbers=left,                    
    numbersep=5pt,                  
    showspaces=false,                
    showstringspaces=false,
    showtabs=false,                  
    tabsize=2,
    postbreak=\%\space,
    language=C
}

%%% sh language codestyle
\lstdefinestyle{shStyle} {
	backgroundcolor=\color{backgroundColour},   
    commentstyle=\color{mGreen},
    keywordstyle=\color{magenta},
    numberstyle=\tiny\color{mGray},
    stringstyle=\color{mPurple},
    basicstyle=\footnotesize,
    breakatwhitespace=false,         
    breaklines=true,                 
    captionpos=b,                    
    keepspaces=true,                 
    numbers=left,                    
    numbersep=5pt,                  
    showspaces=false,                
    showstringspaces=false,
    showtabs=false,                  
    tabsize=2,
    postbreak=\%\space,
    morecomment=[s][\color{red}]{\ -}{\ },
    otherkeywords={$,@,>,<,#,!,ELIF,IF,THEN,ELSE,FI,ESAC,CASE,WHILE,DO,UNTIL,FOR,echo,man,chmod},
    language=sh
}


\lstnewenvironment{CCode}[1]{
	\lstset{style=CStyle}
	\funci{#1}
}{
}

\lstnewenvironment{shCode}[1]{
	\lstset{style=shStyle}
	\funci{#1}
}{
}

\usecolortheme{whale}

\title[СПО]{Системное программное обеспечение}
\subtitle{Введение в язык Си}
\author{Жмылев С. А., Афанасьев Д. Б.}
\date{24.08.2018}

\newenvironment{myenv}[2]{
	\renewcommand{\tabcolsep}{0cm}
	\begin{tabular}{p{0.2\linewidth}p{0.8\linewidth}}
	\textbf{#1} & #2
	\end{tabular}
	\it{}
	}

\newtheorem*{inner}{\innerheader}
\newcommand{\innerheader}{}
\newenvironment{defi}[1]{
		\renewcommand\innerheader{\textbf{#1}}
		\begin{inner}
		\\[0.3em]
		\begin{tabular}{p{0.5cm\linewidth}p{0.8\linewidth}}
		 & \normalfont
		}
		{
		\end{tabular}
		\\
		\end{inner}
		}
		
\newenvironment{important}{\textcolor{red}{Важно: }{}}

\newcommand{\funci}[1]{{\textbf{#1}}}
	
\begin{document}


	\begin{frame}
		\titlepage
	\end{frame}
	 
	\section{Введение в язык Си}
	
	\section{Структура программы}
	\begin{frame}[fragile]
	
		\frametitle{Структура программ}
		
		\begin{CCode}{main.c}
			#include <stdio.h>
			
			void myfunc() {
				printf("Inspiration unlocks the future");
			}
		
			int main(int argc, char *argv[], char *envp[]) { 

				myfunc();			
 				
 				return 0; 
			} \end{CCode}
		
	\end{frame}
	
	\subsection{Аргументы main}
	\begin{frame}[fragile]
		\frametitle{Агрументы main}
		
		\begin{itemize}
		
			\item
			\begin{myenv}{argc}{arguments count}
			Количество переданных программе аргументов, включая имя программы
			\end{myenv}
			
			\item
			\begin{myenv}{*argv[]}{argument vector}
			Массив указателей на каждый из параметров переданных программе аргументов, включая имя программы (агрумент 0)
			\end{myenv}
			
			\item
			\begin{myenv}{*envp[]}{environment parameters}
			Указатель на  массив переменных окружения
			\end{myenv}
			
			% в памяти

			
		\end{itemize}
	
	\end{frame}
	
	\subsection{Заголовочные файлы}
	\begin{frame}[fragile]
	
		\frametitle{Заголовочные файлы}
		
		\begin{defi}{Заголовочный файл}
		Файл, содержимое которого при компилляции автоматически добавляется препроцессором  в исходный текст программы на место директивы
		\end{defi}
		\\ \\

		Может содержать только:
		\begin{itemize}
		
			\item прототипы функций
			\item структуры
			\item союзы
			\item перечисления
			\item макросы
			\item typedef
			\item глобальные переменные
		\end{itemize}	
		
		\begin{important}			
			Не должен содержать описания функций и системных вызовов
		\end{important}
	
	\end{frame}
	
	\subsection{Часто используемые заголовочные файлы UNIX}
	\begin{frame}[fragile]
	
		\frametitle{Часто используемые заголовочные файлы UNIX}
		
		\begin{center}
		\begin{tabular}{l|l}	
			unistd.h 	& объявления UNIX \\
			\hline
			stdio.h  	& стандартный ввод/вывод \\
			\hline
			fcntl.h  	& операции с файлами (например: open) \\
			\hline
			sys/types.h	& системные типы \\
			\hline
			sys/stat.h	& системные статусы \\
			\hline
			errno.h	& errno и и директивы с определением ошибок \\
		\end{tabular}
		\end{center}
		
	\end{frame}
	
	\subsection{Код возврата}
	\begin{frame}[fragile]
	
		\frametitle{Код возврата}
		
		\begin{defi}{Код возврата}
		Число, возвращаемое родительскому процессу при завершении дочернего, позволяющее сделать выводы о результате выполнения программы

		\end{defi}
		
		Как правильнее?
				
		\begin{CCode}{}
			int main(int argc, char *argv[], char *envp[]) { 
 				/* Actions */
 				return 0; 
			} \end{CCode}
		
		или
		
		\begin{CCode}{}
		int main(int argc, char *argv[], char *envp[]) { 
 				/* Actions */
 				return EXIT_SUCCES; 
			} \end{CCode}
		
	\end{frame}
	
	\subsection{Ошибки и стандартизация ошибок: errno.h}
	\begin{frame}[fragile]
	
		\frametitle{Ошибки и стандартизация ошибок: errno.h}
		
		\begin{defi}{errno.h}
		 Заголовочный файл стандартной библиотеки языка программирования С, содержащий: 
		\end{defi}
		
		\begin{itemize}
			\item Переменную \textbf{errno} типа external int
			\item Директивы с определением ошибок и их кодов, например:	
		\end{itemize}
		
		\begin{CCode}{main.c}
			#define ENODEV   19  /* No such device */
			#define ENOTDIR  20  /* Not a directory */
			#define EISDIR   21  /* Is a directory */  \end{CCode}
		
	\end{frame}
	
	\subsection{Ошибки и стандартизация ошибок: perror(3) и strerror(3) }
	\begin{frame}[fragile]
	
		\frametitle{Ошибки и стандартизация ошибок: perror(3) и strerror(3)}
		
		
		\begin{CCode}{perror (3)}
			void perror(const char *str); \end{CCode}
			
		Выводит в STDOUT сообщение вида “str:  описание ошибки”
		
		\\[0.5cm]
		\begin{CCode}{strerror (3)}
			char *strerror(int errnum); \end{CCode}
			
		Возвращает указатель на строку с описанием ошибки
		
	\end{frame}
	
	
	\subsection{Компиляция программы: Этапы компиляции}
	\begin{frame}[fragile]
	
		\frametitle{Компиляция программы: Этапы компиляции}

		\begin{figure}[htbp]
  \centering
  \includegraphics[width=0.15\textwidth]{picpath}
\end{figure}

	\end{frame}
	
	\subsection{Компиляция программы: Чем компилировать}
	\begin{frame}[fragile]
	
		\frametitle{Компиляция программы: Чем компилировать}
		
		\begin{itemize}
		
			\item
			\begin{myenv}{cc}{C compiler}
			\end{myenv}
			
			\item
			\begin{myenv}{gcc}{GNU project C and C++ compiler}
			\end{myenv}
			
			\item
			\begin{myenv}{clang}{C and Objective-C compiler}
			\end{myenv}
			
			% в памяти

			
		\end{itemize}
		
	\end{frame}
	
	\subsection{Компиляция программы: Makefile и утилита make}
	\begin{frame}[fragile]
	
		\frametitle{Компиляция программы: Makefile и утилита make}
		
		\begin{shCode}{Makefile}
		PROJS=main
		CC=gcc
		CFLAGS=-m64

		all: $(PROJS)
			@echo Done!
		$(PROJS):
			$(CC) $(CFLAGS) -o $@ $(@:=.c) \end{shCode}
			
		\begin{shCode}{Утилита make}
		$ make
			gcc -m64 -o main main.c
		Done!
		$ ./main
			Inspiration unlocks the future \end{shCode}
		
	\end{frame}
	
	
 
\end{document}