Многие программы, с которыми вам предстоит работать, работают в каноническом режиме, и терминал по умолчанию работает в этом режиме. Еще раз, каждая строчка попадает в программу после нажатия Enter. Говоря о командном интерпретаторе, надо помнить, что он, в отличие практически от всех остальных, программ работает в неканоническом режиме и читает по одному символу.

После нажатия пользователем клавиши ввода и окончания всех подстановок командный интерпретатор производит синтаксический разбор введенной строки и разделяет ее на отдельные слова с помощью специальных разделителей, хранимых в переменной IFS.

Если вы не устанавливаете значение IFS, то оно по умолчанию содержит пробел, табуляцию и перенос на новую строку. Как уже сказано, в IFS может храниться несколько значений.

\begin{shCode}{Например}
		ag@helios:/home/ag$ echo    123
		123 \end{shCode}
Проигнорирует пробелы перед строкой 123.

%Однако shell несколько умнее, поэтому в некоторых случаях пробел не будет интерпретироваться как IFS.