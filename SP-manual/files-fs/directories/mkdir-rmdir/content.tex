Для создания директорий есть системный вызов mkdir (2). Он принимает путь к новой директории и режим доступа mode.

\begin{CCode}{mkdir(2)}
	int mkdir(
		const char *path,	/* path to directory */
		mode_t mode 		/* access mode */
	); \end{CCode}

mode\_t -- это режим доступа, который будет установлен на директорию. Этот режим доступа складывается с umask и маской родительской директории.

Системный вызов rmdir (2) принимает путь и удаляет директорию в случае, если директория имеет всего две жестких ссылки, это точка и две точки. Если внутри директории больше жестких ссылок, то директория не удалится, поскольку она не пустая с точки зрения ОС.

\begin{CCode}{rmdir(2)}
	int rmdir(
		const char *path /* path to directory */
	); \end{CCode}

Оба вызова возвращают 0 или код ошибки.