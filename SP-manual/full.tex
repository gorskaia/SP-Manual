\documentclass[oneside]{book}

% ------------------------------
% Table of contents

\setcounter{secnumdepth}{4}
\usepackage[pdftex]{hyperref}

\usepackage{natbib}

% ------------------------------
% Languages & fonts

\usepackage[utf8]{inputenc}
\usepackage[english,russian]{babel}

\usepackage{amsmath,amsthm,amssymb}
\usepackage{mathtext}
\usepackage[T1,T2A]{fontenc}

\renewcommand*{\familydefault}{\sfdefault}
\usepackage{droid}

\usepackage{microtype}
\sloppy

% ------------------------------
% Page geometry & text spacing

\usepackage[left=2.5cm,right=2.5cm,top=4cm,bottom=4cm]{geometry}

\usepackage{parskip}
\setlength{\parindent}{0em}
\setlength{\parskip}{1em}


\usepackage{atbegshi}% http://ctan.org/pkg/atbegshi
\AtBeginDocument{\AtBeginShipoutNext{\AtBeginShipoutDiscard}}

\let\OLDitemize\itemize
\renewcommand\itemize{\OLDitemize\addtolength{\itemsep}{-5pt}}


\renewcommand{\arraystretch}{1.5}

\usepackage{graphicx}

% ------------------------------
% Code listings

\usepackage{xcolor}
\usepackage{listings}

\definecolor{mGreen}{rgb}{0,0.6,0}
\definecolor{mGray}{rgb}{0.5,0.5,0.5}
\definecolor{mPurple}{rgb}{0.58,0,0.82}
\definecolor{backgroundColour}{rgb}{0.95,0.96,0.97}


%%% C language codestyle
\lstdefinestyle{CStyle}{
    backgroundcolor=\color{backgroundColour},   
    commentstyle=\color{mGreen},
    keywordstyle=\color{blue},
    numberstyle=\tiny\color{mGray},
    stringstyle=\color{mPurple},
    basicstyle=\footnotesize,
    breakatwhitespace=false,         
    breaklines=true,                 
    captionpos=b,                    
    keepspaces=true,                 
    numbers=left,                    
    numbersep=5pt,                  
    showspaces=false,                
    showstringspaces=false,
    showtabs=false,                  
    tabsize=2,
    postbreak=\%\space,
    belowskip=-1,
    language=C
}

%%% sh language codestyle
\lstdefinestyle{shStyle} {
	backgroundcolor=\color{backgroundColour},   
    commentstyle=\color{mGreen},
    keywordstyle=\color{magenta},
    numberstyle=\tiny\color{mGray},
    stringstyle=\color{mPurple},
    basicstyle=\footnotesize,
    breakatwhitespace=false,         
    breaklines=true,                 
    captionpos=b,                    
    keepspaces=true,                 
    numbers=left,                    
    numbersep=5pt,                  
    showspaces=false,                
    showstringspaces=false,
    showtabs=false,                  
    tabsize=2,
    postbreak=\%\space,
    morecomment=[s][\color{red}]{\ -}{\ },
    otherkeywords={$,@,>,<,#,!,ELIF,IF,THEN,ELSE,FI,ESAC,CASE,WHILE,DO,UNTIL,FOR,echo,man,chmod},
    language=sh
}


\lstnewenvironment{CCode}[1]{
	\lstset{style=CStyle}
	\funci{#1}~}{
}

\lstnewenvironment{shCode}[1]{
	\lstset{style=shStyle}
	\funci{#1}~}{
}

% ------------------------------
% Pictures

\usepackage{graphicx}

% ------------------------------
% Special commands an environments

\newenvironment{myenv}[2]{
	\renewcommand{\tabcolsep}{0cm}
	\begin{tabular}{p{0.15\linewidth}p{0.85\linewidth}}
	\textbf{#1} & #2
	\end{tabular}\\[0.5em]
	}{
	}
	
	
%%% Definitions

\usepackage{xpatch}5
\makeatletter
\xpatchcmd{\@thm}{\thm@headpunct{.}}{\thm@headpunct{ --- \\}}{}{}
\makeatother

\newenvironment{defi}[1]{
		\textbf{#1} ---
		\\[0.3em]
		\begin{tabular}{p{0.5cm\linewidth}p{0.9\linewidth}}
			 & 
		}
		{
		\end{tabular}
		}

%%% Important sentences
	
\newenvironment{important}{\textbf{Важно: }{}}
		
\newenvironment{boxy}{
\renewcommand{\arraystretch}{1.8}
	\begin{tabular}{|p{1\linewidth}|}
	\hline
	}
	{
	\\ \hline
	\end{tabular}
	}

\newenvironment{funci}[1]{{{\textbf{#1}}}}{}



\begin{document}

\cleardoublepage\clearpage

\begin{titlepage}
	% \centering	
	\Large ПРОЕКТ УЧЕБНОГО ПОСОБИЯ
	
	\vspace {4cm}
	
	\Huge\textbf{{\fontsize{30}{30}\selectfont 
	СИСТЕМНОЕ ПРОГРАММНОЕ
	\\
	ОБЕСПЕЧЕНИЕ}}
	
	\vspace{0.5cm}
	{\large Операционная система UNIX, интерпретатор shell, язык C \par}
	


	\vfill
	{\large НИУ ИТМО, 2018}
	
\end{titlepage}

\tableofcontents

% ------------------------------
% Часть: Основы системного программирования

\part{Системное программное обеспечение}
%Именованные каналы во многом работают так же, как и неименованные каналы, но все же имеют несколько заметных отличий.

\begin{itemize}
	\item именованные каналы существуют в виде специального файла устройства в файловой системе;
	\item процессы различного происхождения могут разделять данные через такой канал;
	\item именованный канал остается в файловой системе для дальнейшего использования и после того, как весь ввод/вывод сделан.
\end{itemize}


Существует два способа создания именованного канала:

Создать обычный файл, директорию или файл специального назначения с помощью системного вызова mknode (2), указав 0 в dev_t.

\begin{CCode}{mknode (2)}
	int mknod(
		const char *path, 
		mode_t mode, 
		dev_t dev
	); \end{CCode}

или воспользоваться функцией mkfifo (3)

\begin{CCode}{mkfifo (3)}
	int mkfifo(
		const char *pathname, 
		mode_t mode
	); \end{CCode}


	\chapter{Операционная система}
	Именованные каналы во многом работают так же, как и неименованные каналы, но все же имеют несколько заметных отличий.

\begin{itemize}
	\item именованные каналы существуют в виде специального файла устройства в файловой системе;
	\item процессы различного происхождения могут разделять данные через такой канал;
	\item именованный канал остается в файловой системе для дальнейшего использования и после того, как весь ввод/вывод сделан.
\end{itemize}


Существует два способа создания именованного канала:

Создать обычный файл, директорию или файл специального назначения с помощью системного вызова mknode (2), указав 0 в dev_t.

\begin{CCode}{mknode (2)}
	int mknod(
		const char *path, 
		mode_t mode, 
		dev_t dev
	); \end{CCode}

или воспользоваться функцией mkfifo (3)

\begin{CCode}{mkfifo (3)}
	int mkfifo(
		const char *pathname, 
		mode_t mode
	); \end{CCode}

			
		\section{Роль операционной системы}
		Именованные каналы во многом работают так же, как и неименованные каналы, но все же имеют несколько заметных отличий.

\begin{itemize}
	\item именованные каналы существуют в виде специального файла устройства в файловой системе;
	\item процессы различного происхождения могут разделять данные через такой канал;
	\item именованный канал остается в файловой системе для дальнейшего использования и после того, как весь ввод/вывод сделан.
\end{itemize}


Существует два способа создания именованного канала:

Создать обычный файл, директорию или файл специального назначения с помощью системного вызова mknode (2), указав 0 в dev_t.

\begin{CCode}{mknode (2)}
	int mknod(
		const char *path, 
		mode_t mode, 
		dev_t dev
	); \end{CCode}

или воспользоваться функцией mkfifo (3)

\begin{CCode}{mkfifo (3)}
	int mkfifo(
		const char *pathname, 
		mode_t mode
	); \end{CCode}

	
		\section{Типовая ОС UNIX}
		Именованные каналы во многом работают так же, как и неименованные каналы, но все же имеют несколько заметных отличий.

\begin{itemize}
	\item именованные каналы существуют в виде специального файла устройства в файловой системе;
	\item процессы различного происхождения могут разделять данные через такой канал;
	\item именованный канал остается в файловой системе для дальнейшего использования и после того, как весь ввод/вывод сделан.
\end{itemize}


Существует два способа создания именованного канала:

Создать обычный файл, директорию или файл специального назначения с помощью системного вызова mknode (2), указав 0 в dev_t.

\begin{CCode}{mknode (2)}
	int mknod(
		const char *path, 
		mode_t mode, 
		dev_t dev
	); \end{CCode}

или воспользоваться функцией mkfifo (3)

\begin{CCode}{mkfifo (3)}
	int mkfifo(
		const char *pathname, 
		mode_t mode
	); \end{CCode}

	
		\section{Ядро типовой ОС UNIX}
		Именованные каналы во многом работают так же, как и неименованные каналы, но все же имеют несколько заметных отличий.

\begin{itemize}
	\item именованные каналы существуют в виде специального файла устройства в файловой системе;
	\item процессы различного происхождения могут разделять данные через такой канал;
	\item именованный канал остается в файловой системе для дальнейшего использования и после того, как весь ввод/вывод сделан.
\end{itemize}


Существует два способа создания именованного канала:

Создать обычный файл, директорию или файл специального назначения с помощью системного вызова mknode (2), указав 0 в dev_t.

\begin{CCode}{mknode (2)}
	int mknod(
		const char *path, 
		mode_t mode, 
		dev_t dev
	); \end{CCode}

или воспользоваться функцией mkfifo (3)

\begin{CCode}{mkfifo (3)}
	int mkfifo(
		const char *pathname, 
		mode_t mode
	); \end{CCode}

		
			\subsection{Подсистемы ядра}
			Именованные каналы во многом работают так же, как и неименованные каналы, но все же имеют несколько заметных отличий.

\begin{itemize}
	\item именованные каналы существуют в виде специального файла устройства в файловой системе;
	\item процессы различного происхождения могут разделять данные через такой канал;
	\item именованный канал остается в файловой системе для дальнейшего использования и после того, как весь ввод/вывод сделан.
\end{itemize}


Существует два способа создания именованного канала:

Создать обычный файл, директорию или файл специального назначения с помощью системного вызова mknode (2), указав 0 в dev_t.

\begin{CCode}{mknode (2)}
	int mknod(
		const char *path, 
		mode_t mode, 
		dev_t dev
	); \end{CCode}

или воспользоваться функцией mkfifo (3)

\begin{CCode}{mkfifo (3)}
	int mkfifo(
		const char *pathname, 
		mode_t mode
	); \end{CCode}

		
			\subsection{Исключения, прерывания, системные вызовы}
			Именованные каналы во многом работают так же, как и неименованные каналы, но все же имеют несколько заметных отличий.

\begin{itemize}
	\item именованные каналы существуют в виде специального файла устройства в файловой системе;
	\item процессы различного происхождения могут разделять данные через такой канал;
	\item именованный канал остается в файловой системе для дальнейшего использования и после того, как весь ввод/вывод сделан.
\end{itemize}


Существует два способа создания именованного канала:

Создать обычный файл, директорию или файл специального назначения с помощью системного вызова mknode (2), указав 0 в dev_t.

\begin{CCode}{mknode (2)}
	int mknod(
		const char *path, 
		mode_t mode, 
		dev_t dev
	); \end{CCode}

или воспользоваться функцией mkfifo (3)

\begin{CCode}{mkfifo (3)}
	int mkfifo(
		const char *pathname, 
		mode_t mode
	); \end{CCode}

			
				\subsubsection{Исключения}
				Именованные каналы во многом работают так же, как и неименованные каналы, но все же имеют несколько заметных отличий.

\begin{itemize}
	\item именованные каналы существуют в виде специального файла устройства в файловой системе;
	\item процессы различного происхождения могут разделять данные через такой канал;
	\item именованный канал остается в файловой системе для дальнейшего использования и после того, как весь ввод/вывод сделан.
\end{itemize}


Существует два способа создания именованного канала:

Создать обычный файл, директорию или файл специального назначения с помощью системного вызова mknode (2), указав 0 в dev_t.

\begin{CCode}{mknode (2)}
	int mknod(
		const char *path, 
		mode_t mode, 
		dev_t dev
	); \end{CCode}

или воспользоваться функцией mkfifo (3)

\begin{CCode}{mkfifo (3)}
	int mkfifo(
		const char *pathname, 
		mode_t mode
	); \end{CCode}

				
				\subsubsection{Прерывания}
				Именованные каналы во многом работают так же, как и неименованные каналы, но все же имеют несколько заметных отличий.

\begin{itemize}
	\item именованные каналы существуют в виде специального файла устройства в файловой системе;
	\item процессы различного происхождения могут разделять данные через такой канал;
	\item именованный канал остается в файловой системе для дальнейшего использования и после того, как весь ввод/вывод сделан.
\end{itemize}


Существует два способа создания именованного канала:

Создать обычный файл, директорию или файл специального назначения с помощью системного вызова mknode (2), указав 0 в dev_t.

\begin{CCode}{mknode (2)}
	int mknod(
		const char *path, 
		mode_t mode, 
		dev_t dev
	); \end{CCode}

или воспользоваться функцией mkfifo (3)

\begin{CCode}{mkfifo (3)}
	int mkfifo(
		const char *pathname, 
		mode_t mode
	); \end{CCode}

				
				\subsubsection{Системные вызовы.}
				Именованные каналы во многом работают так же, как и неименованные каналы, но все же имеют несколько заметных отличий.

\begin{itemize}
	\item именованные каналы существуют в виде специального файла устройства в файловой системе;
	\item процессы различного происхождения могут разделять данные через такой канал;
	\item именованный канал остается в файловой системе для дальнейшего использования и после того, как весь ввод/вывод сделан.
\end{itemize}


Существует два способа создания именованного канала:

Создать обычный файл, директорию или файл специального назначения с помощью системного вызова mknode (2), указав 0 в dev_t.

\begin{CCode}{mknode (2)}
	int mknod(
		const char *path, 
		mode_t mode, 
		dev_t dev
	); \end{CCode}

или воспользоваться функцией mkfifo (3)

\begin{CCode}{mkfifo (3)}
	int mkfifo(
		const char *pathname, 
		mode_t mode
	); \end{CCode}

		
		
	\chapter{Концепция системного программирования}
	Именованные каналы во многом работают так же, как и неименованные каналы, но все же имеют несколько заметных отличий.

\begin{itemize}
	\item именованные каналы существуют в виде специального файла устройства в файловой системе;
	\item процессы различного происхождения могут разделять данные через такой канал;
	\item именованный канал остается в файловой системе для дальнейшего использования и после того, как весь ввод/вывод сделан.
\end{itemize}


Существует два способа создания именованного канала:

Создать обычный файл, директорию или файл специального назначения с помощью системного вызова mknode (2), указав 0 в dev_t.

\begin{CCode}{mknode (2)}
	int mknod(
		const char *path, 
		mode_t mode, 
		dev_t dev
	); \end{CCode}

или воспользоваться функцией mkfifo (3)

\begin{CCode}{mkfifo (3)}
	int mkfifo(
		const char *pathname, 
		mode_t mode
	); \end{CCode}

		
		\section{Языки системного программирования}
		Именованные каналы во многом работают так же, как и неименованные каналы, но все же имеют несколько заметных отличий.

\begin{itemize}
	\item именованные каналы существуют в виде специального файла устройства в файловой системе;
	\item процессы различного происхождения могут разделять данные через такой канал;
	\item именованный канал остается в файловой системе для дальнейшего использования и после того, как весь ввод/вывод сделан.
\end{itemize}


Существует два способа создания именованного канала:

Создать обычный файл, директорию или файл специального назначения с помощью системного вызова mknode (2), указав 0 в dev_t.

\begin{CCode}{mknode (2)}
	int mknod(
		const char *path, 
		mode_t mode, 
		dev_t dev
	); \end{CCode}

или воспользоваться функцией mkfifo (3)

\begin{CCode}{mkfifo (3)}
	int mkfifo(
		const char *pathname, 
		mode_t mode
	); \end{CCode}

		
		\section{Системные вызовы}
		Именованные каналы во многом работают так же, как и неименованные каналы, но все же имеют несколько заметных отличий.

\begin{itemize}
	\item именованные каналы существуют в виде специального файла устройства в файловой системе;
	\item процессы различного происхождения могут разделять данные через такой канал;
	\item именованный канал остается в файловой системе для дальнейшего использования и после того, как весь ввод/вывод сделан.
\end{itemize}


Существует два способа создания именованного канала:

Создать обычный файл, директорию или файл специального назначения с помощью системного вызова mknode (2), указав 0 в dev_t.

\begin{CCode}{mknode (2)}
	int mknod(
		const char *path, 
		mode_t mode, 
		dev_t dev
	); \end{CCode}

или воспользоваться функцией mkfifo (3)

\begin{CCode}{mkfifo (3)}
	int mkfifo(
		const char *pathname, 
		mode_t mode
	); \end{CCode}

		
		\section{Ввод-вывод}
		Именованные каналы во многом работают так же, как и неименованные каналы, но все же имеют несколько заметных отличий.

\begin{itemize}
	\item именованные каналы существуют в виде специального файла устройства в файловой системе;
	\item процессы различного происхождения могут разделять данные через такой канал;
	\item именованный канал остается в файловой системе для дальнейшего использования и после того, как весь ввод/вывод сделан.
\end{itemize}


Существует два способа создания именованного канала:

Создать обычный файл, директорию или файл специального назначения с помощью системного вызова mknode (2), указав 0 в dev_t.

\begin{CCode}{mknode (2)}
	int mknod(
		const char *path, 
		mode_t mode, 
		dev_t dev
	); \end{CCode}

или воспользоваться функцией mkfifo (3)

\begin{CCode}{mkfifo (3)}
	int mkfifo(
		const char *pathname, 
		mode_t mode
	); \end{CCode}

		
		\section{Процессы и потоки}
		Именованные каналы во многом работают так же, как и неименованные каналы, но все же имеют несколько заметных отличий.

\begin{itemize}
	\item именованные каналы существуют в виде специального файла устройства в файловой системе;
	\item процессы различного происхождения могут разделять данные через такой канал;
	\item именованный канал остается в файловой системе для дальнейшего использования и после того, как весь ввод/вывод сделан.
\end{itemize}


Существует два способа создания именованного канала:

Создать обычный файл, директорию или файл специального назначения с помощью системного вызова mknode (2), указав 0 в dev_t.

\begin{CCode}{mknode (2)}
	int mknod(
		const char *path, 
		mode_t mode, 
		dev_t dev
	); \end{CCode}

или воспользоваться функцией mkfifo (3)

\begin{CCode}{mkfifo (3)}
	int mkfifo(
		const char *pathname, 
		mode_t mode
	); \end{CCode}

	
	\chapter{Работа в ОС UNIX}
	
		\section{Терминал}
		Именованные каналы во многом работают так же, как и неименованные каналы, но все же имеют несколько заметных отличий.

\begin{itemize}
	\item именованные каналы существуют в виде специального файла устройства в файловой системе;
	\item процессы различного происхождения могут разделять данные через такой канал;
	\item именованный канал остается в файловой системе для дальнейшего использования и после того, как весь ввод/вывод сделан.
\end{itemize}


Существует два способа создания именованного канала:

Создать обычный файл, директорию или файл специального назначения с помощью системного вызова mknode (2), указав 0 в dev_t.

\begin{CCode}{mknode (2)}
	int mknod(
		const char *path, 
		mode_t mode, 
		dev_t dev
	); \end{CCode}

или воспользоваться функцией mkfifo (3)

\begin{CCode}{mkfifo (3)}
	int mkfifo(
		const char *pathname, 
		mode_t mode
	); \end{CCode}

			
			\subsection{Канонический и неканонический режимы ввода}
			Именованные каналы во многом работают так же, как и неименованные каналы, но все же имеют несколько заметных отличий.

\begin{itemize}
	\item именованные каналы существуют в виде специального файла устройства в файловой системе;
	\item процессы различного происхождения могут разделять данные через такой канал;
	\item именованный канал остается в файловой системе для дальнейшего использования и после того, как весь ввод/вывод сделан.
\end{itemize}


Существует два способа создания именованного канала:

Создать обычный файл, директорию или файл специального назначения с помощью системного вызова mknode (2), указав 0 в dev_t.

\begin{CCode}{mknode (2)}
	int mknod(
		const char *path, 
		mode_t mode, 
		dev_t dev
	); \end{CCode}

или воспользоваться функцией mkfifo (3)

\begin{CCode}{mkfifo (3)}
	int mkfifo(
		const char *pathname, 
		mode_t mode
	); \end{CCode}

		
		\section{Командный интерпретатор}
		Именованные каналы во многом работают так же, как и неименованные каналы, но все же имеют несколько заметных отличий.

\begin{itemize}
	\item именованные каналы существуют в виде специального файла устройства в файловой системе;
	\item процессы различного происхождения могут разделять данные через такой канал;
	\item именованный канал остается в файловой системе для дальнейшего использования и после того, как весь ввод/вывод сделан.
\end{itemize}


Существует два способа создания именованного канала:

Создать обычный файл, директорию или файл специального назначения с помощью системного вызова mknode (2), указав 0 в dev_t.

\begin{CCode}{mknode (2)}
	int mknod(
		const char *path, 
		mode_t mode, 
		dev_t dev
	); \end{CCode}

или воспользоваться функцией mkfifo (3)

\begin{CCode}{mkfifo (3)}
	int mkfifo(
		const char *pathname, 
		mode_t mode
	); \end{CCode}

		
		\section{Программа, утилита, команда}
		Именованные каналы во многом работают так же, как и неименованные каналы, но все же имеют несколько заметных отличий.

\begin{itemize}
	\item именованные каналы существуют в виде специального файла устройства в файловой системе;
	\item процессы различного происхождения могут разделять данные через такой канал;
	\item именованный канал остается в файловой системе для дальнейшего использования и после того, как весь ввод/вывод сделан.
\end{itemize}


Существует два способа создания именованного канала:

Создать обычный файл, директорию или файл специального назначения с помощью системного вызова mknode (2), указав 0 в dev_t.

\begin{CCode}{mknode (2)}
	int mknod(
		const char *path, 
		mode_t mode, 
		dev_t dev
	); \end{CCode}

или воспользоваться функцией mkfifo (3)

\begin{CCode}{mkfifo (3)}
	int mkfifo(
		const char *pathname, 
		mode_t mode
	); \end{CCode}



% ------------------------------
% Часть: Командный интерпретарор shell

\part{Командный интерпретатор shell}
%Именованные каналы во многом работают так же, как и неименованные каналы, но все же имеют несколько заметных отличий.

\begin{itemize}
	\item именованные каналы существуют в виде специального файла устройства в файловой системе;
	\item процессы различного происхождения могут разделять данные через такой канал;
	\item именованный канал остается в файловой системе для дальнейшего использования и после того, как весь ввод/вывод сделан.
\end{itemize}


Существует два способа создания именованного канала:

Создать обычный файл, директорию или файл специального назначения с помощью системного вызова mknode (2), указав 0 в dev_t.

\begin{CCode}{mknode (2)}
	int mknod(
		const char *path, 
		mode_t mode, 
		dev_t dev
	); \end{CCode}

или воспользоваться функцией mkfifo (3)

\begin{CCode}{mkfifo (3)}
	int mkfifo(
		const char *pathname, 
		mode_t mode
	); \end{CCode}


	\chapter{Основные понятия}
	Именованные каналы во многом работают так же, как и неименованные каналы, но все же имеют несколько заметных отличий.

\begin{itemize}
	\item именованные каналы существуют в виде специального файла устройства в файловой системе;
	\item процессы различного происхождения могут разделять данные через такой канал;
	\item именованный канал остается в файловой системе для дальнейшего использования и после того, как весь ввод/вывод сделан.
\end{itemize}


Существует два способа создания именованного канала:

Создать обычный файл, директорию или файл специального назначения с помощью системного вызова mknode (2), указав 0 в dev_t.

\begin{CCode}{mknode (2)}
	int mknod(
		const char *path, 
		mode_t mode, 
		dev_t dev
	); \end{CCode}

или воспользоваться функцией mkfifo (3)

\begin{CCode}{mkfifo (3)}
	int mkfifo(
		const char *pathname, 
		mode_t mode
	); \end{CCode}

		
		\section{Командный интерпретатор}
		Именованные каналы во многом работают так же, как и неименованные каналы, но все же имеют несколько заметных отличий.

\begin{itemize}
	\item именованные каналы существуют в виде специального файла устройства в файловой системе;
	\item процессы различного происхождения могут разделять данные через такой канал;
	\item именованный канал остается в файловой системе для дальнейшего использования и после того, как весь ввод/вывод сделан.
\end{itemize}


Существует два способа создания именованного канала:

Создать обычный файл, директорию или файл специального назначения с помощью системного вызова mknode (2), указав 0 в dev_t.

\begin{CCode}{mknode (2)}
	int mknod(
		const char *path, 
		mode_t mode, 
		dev_t dev
	); \end{CCode}

или воспользоваться функцией mkfifo (3)

\begin{CCode}{mkfifo (3)}
	int mkfifo(
		const char *pathname, 
		mode_t mode
	); \end{CCode}

		
			\subsection{Приглашения командной строки (prompt string)}
			Именованные каналы во многом работают так же, как и неименованные каналы, но все же имеют несколько заметных отличий.

\begin{itemize}
	\item именованные каналы существуют в виде специального файла устройства в файловой системе;
	\item процессы различного происхождения могут разделять данные через такой канал;
	\item именованный канал остается в файловой системе для дальнейшего использования и после того, как весь ввод/вывод сделан.
\end{itemize}


Существует два способа создания именованного канала:

Создать обычный файл, директорию или файл специального назначения с помощью системного вызова mknode (2), указав 0 в dev_t.

\begin{CCode}{mknode (2)}
	int mknod(
		const char *path, 
		mode_t mode, 
		dev_t dev
	); \end{CCode}

или воспользоваться функцией mkfifo (3)

\begin{CCode}{mkfifo (3)}
	int mkfifo(
		const char *pathname, 
		mode_t mode
	); \end{CCode}
			
					
		\section{Принципы исполнения инструкций}
		
			\subsection{IFS}
			Именованные каналы во многом работают так же, как и неименованные каналы, но все же имеют несколько заметных отличий.

\begin{itemize}
	\item именованные каналы существуют в виде специального файла устройства в файловой системе;
	\item процессы различного происхождения могут разделять данные через такой канал;
	\item именованный канал остается в файловой системе для дальнейшего использования и после того, как весь ввод/вывод сделан.
\end{itemize}


Существует два способа создания именованного канала:

Создать обычный файл, директорию или файл специального назначения с помощью системного вызова mknode (2), указав 0 в dev_t.

\begin{CCode}{mknode (2)}
	int mknod(
		const char *path, 
		mode_t mode, 
		dev_t dev
	); \end{CCode}

или воспользоваться функцией mkfifo (3)

\begin{CCode}{mkfifo (3)}
	int mkfifo(
		const char *pathname, 
		mode_t mode
	); \end{CCode}

			
			\subsection{Группировка команд и запуск команд в subshell}
			Именованные каналы во многом работают так же, как и неименованные каналы, но все же имеют несколько заметных отличий.

\begin{itemize}
	\item именованные каналы существуют в виде специального файла устройства в файловой системе;
	\item процессы различного происхождения могут разделять данные через такой канал;
	\item именованный канал остается в файловой системе для дальнейшего использования и после того, как весь ввод/вывод сделан.
\end{itemize}


Существует два способа создания именованного канала:

Создать обычный файл, директорию или файл специального назначения с помощью системного вызова mknode (2), указав 0 в dev_t.

\begin{CCode}{mknode (2)}
	int mknod(
		const char *path, 
		mode_t mode, 
		dev_t dev
	); \end{CCode}

или воспользоваться функцией mkfifo (3)

\begin{CCode}{mkfifo (3)}
	int mkfifo(
		const char *pathname, 
		mode_t mode
	); \end{CCode}

			
			\subsection{Разделители команд}
			Именованные каналы во многом работают так же, как и неименованные каналы, но все же имеют несколько заметных отличий.

\begin{itemize}
	\item именованные каналы существуют в виде специального файла устройства в файловой системе;
	\item процессы различного происхождения могут разделять данные через такой канал;
	\item именованный канал остается в файловой системе для дальнейшего использования и после того, как весь ввод/вывод сделан.
\end{itemize}


Существует два способа создания именованного канала:

Создать обычный файл, директорию или файл специального назначения с помощью системного вызова mknode (2), указав 0 в dev_t.

\begin{CCode}{mknode (2)}
	int mknod(
		const char *path, 
		mode_t mode, 
		dev_t dev
	); \end{CCode}

или воспользоваться функцией mkfifo (3)

\begin{CCode}{mkfifo (3)}
	int mkfifo(
		const char *pathname, 
		mode_t mode
	); \end{CCode}

			
			\subsection{Glob-джокеры}
			Именованные каналы во многом работают так же, как и неименованные каналы, но все же имеют несколько заметных отличий.

\begin{itemize}
	\item именованные каналы существуют в виде специального файла устройства в файловой системе;
	\item процессы различного происхождения могут разделять данные через такой канал;
	\item именованный канал остается в файловой системе для дальнейшего использования и после того, как весь ввод/вывод сделан.
\end{itemize}


Существует два способа создания именованного канала:

Создать обычный файл, директорию или файл специального назначения с помощью системного вызова mknode (2), указав 0 в dev_t.

\begin{CCode}{mknode (2)}
	int mknod(
		const char *path, 
		mode_t mode, 
		dev_t dev
	); \end{CCode}

или воспользоваться функцией mkfifo (3)

\begin{CCode}{mkfifo (3)}
	int mkfifo(
		const char *pathname, 
		mode_t mode
	); \end{CCode}

			
			\subsection{Переменные}
			Именованные каналы во многом работают так же, как и неименованные каналы, но все же имеют несколько заметных отличий.

\begin{itemize}
	\item именованные каналы существуют в виде специального файла устройства в файловой системе;
	\item процессы различного происхождения могут разделять данные через такой канал;
	\item именованный канал остается в файловой системе для дальнейшего использования и после того, как весь ввод/вывод сделан.
\end{itemize}


Существует два способа создания именованного канала:

Создать обычный файл, директорию или файл специального назначения с помощью системного вызова mknode (2), указав 0 в dev_t.

\begin{CCode}{mknode (2)}
	int mknod(
		const char *path, 
		mode_t mode, 
		dev_t dev
	); \end{CCode}

или воспользоваться функцией mkfifo (3)

\begin{CCode}{mkfifo (3)}
	int mkfifo(
		const char *pathname, 
		mode_t mode
	); \end{CCode}

						
				\subsubsection{Обрезка переменных}
				Именованные каналы во многом работают так же, как и неименованные каналы, но все же имеют несколько заметных отличий.

\begin{itemize}
	\item именованные каналы существуют в виде специального файла устройства в файловой системе;
	\item процессы различного происхождения могут разделять данные через такой канал;
	\item именованный канал остается в файловой системе для дальнейшего использования и после того, как весь ввод/вывод сделан.
\end{itemize}


Существует два способа создания именованного канала:

Создать обычный файл, директорию или файл специального назначения с помощью системного вызова mknode (2), указав 0 в dev_t.

\begin{CCode}{mknode (2)}
	int mknod(
		const char *path, 
		mode_t mode, 
		dev_t dev
	); \end{CCode}

или воспользоваться функцией mkfifo (3)

\begin{CCode}{mkfifo (3)}
	int mkfifo(
		const char *pathname, 
		mode_t mode
	); \end{CCode}

			
			\subsection{Как shell интерпретирует команду}
			Именованные каналы во многом работают так же, как и неименованные каналы, но все же имеют несколько заметных отличий.

\begin{itemize}
	\item именованные каналы существуют в виде специального файла устройства в файловой системе;
	\item процессы различного происхождения могут разделять данные через такой канал;
	\item именованный канал остается в файловой системе для дальнейшего использования и после того, как весь ввод/вывод сделан.
\end{itemize}


Существует два способа создания именованного канала:

Создать обычный файл, директорию или файл специального назначения с помощью системного вызова mknode (2), указав 0 в dev_t.

\begin{CCode}{mknode (2)}
	int mknod(
		const char *path, 
		mode_t mode, 
		dev_t dev
	); \end{CCode}

или воспользоваться функцией mkfifo (3)

\begin{CCode}{mkfifo (3)}
	int mkfifo(
		const char *pathname, 
		mode_t mode
	); \end{CCode}

		
	\chapter{Скрипты shell}
	Именованные каналы во многом работают так же, как и неименованные каналы, но все же имеют несколько заметных отличий.

\begin{itemize}
	\item именованные каналы существуют в виде специального файла устройства в файловой системе;
	\item процессы различного происхождения могут разделять данные через такой канал;
	\item именованный канал остается в файловой системе для дальнейшего использования и после того, как весь ввод/вывод сделан.
\end{itemize}


Существует два способа создания именованного канала:

Создать обычный файл, директорию или файл специального назначения с помощью системного вызова mknode (2), указав 0 в dev_t.

\begin{CCode}{mknode (2)}
	int mknod(
		const char *path, 
		mode_t mode, 
		dev_t dev
	); \end{CCode}

или воспользоваться функцией mkfifo (3)

\begin{CCode}{mkfifo (3)}
	int mkfifo(
		const char *pathname, 
		mode_t mode
	); \end{CCode}

	
		\section{Именование и запуск скриптов}
		Именованные каналы во многом работают так же, как и неименованные каналы, но все же имеют несколько заметных отличий.

\begin{itemize}
	\item именованные каналы существуют в виде специального файла устройства в файловой системе;
	\item процессы различного происхождения могут разделять данные через такой канал;
	\item именованный канал остается в файловой системе для дальнейшего использования и после того, как весь ввод/вывод сделан.
\end{itemize}


Существует два способа создания именованного канала:

Создать обычный файл, директорию или файл специального назначения с помощью системного вызова mknode (2), указав 0 в dev_t.

\begin{CCode}{mknode (2)}
	int mknod(
		const char *path, 
		mode_t mode, 
		dev_t dev
	); \end{CCode}

или воспользоваться функцией mkfifo (3)

\begin{CCode}{mkfifo (3)}
	int mkfifo(
		const char *pathname, 
		mode_t mode
	); \end{CCode}

		
		\section{Позиционные параметры}
		Именованные каналы во многом работают так же, как и неименованные каналы, но все же имеют несколько заметных отличий.

\begin{itemize}
	\item именованные каналы существуют в виде специального файла устройства в файловой системе;
	\item процессы различного происхождения могут разделять данные через такой канал;
	\item именованный канал остается в файловой системе для дальнейшего использования и после того, как весь ввод/вывод сделан.
\end{itemize}


Существует два способа создания именованного канала:

Создать обычный файл, директорию или файл специального назначения с помощью системного вызова mknode (2), указав 0 в dev_t.

\begin{CCode}{mknode (2)}
	int mknod(
		const char *path, 
		mode_t mode, 
		dev_t dev
	); \end{CCode}

или воспользоваться функцией mkfifo (3)

\begin{CCode}{mkfifo (3)}
	int mkfifo(
		const char *pathname, 
		mode_t mode
	); \end{CCode}

		
		\section{Специальные переменные}
		Именованные каналы во многом работают так же, как и неименованные каналы, но все же имеют несколько заметных отличий.

\begin{itemize}
	\item именованные каналы существуют в виде специального файла устройства в файловой системе;
	\item процессы различного происхождения могут разделять данные через такой канал;
	\item именованный канал остается в файловой системе для дальнейшего использования и после того, как весь ввод/вывод сделан.
\end{itemize}


Существует два способа создания именованного канала:

Создать обычный файл, директорию или файл специального назначения с помощью системного вызова mknode (2), указав 0 в dev_t.

\begin{CCode}{mknode (2)}
	int mknod(
		const char *path, 
		mode_t mode, 
		dev_t dev
	); \end{CCode}

или воспользоваться функцией mkfifo (3)

\begin{CCode}{mkfifo (3)}
	int mkfifo(
		const char *pathname, 
		mode_t mode
	); \end{CCode}

		
	\chapter{Встроенные возможности и конструкции shell}
	Именованные каналы во многом работают так же, как и неименованные каналы, но все же имеют несколько заметных отличий.

\begin{itemize}
	\item именованные каналы существуют в виде специального файла устройства в файловой системе;
	\item процессы различного происхождения могут разделять данные через такой канал;
	\item именованный канал остается в файловой системе для дальнейшего использования и после того, как весь ввод/вывод сделан.
\end{itemize}


Существует два способа создания именованного канала:

Создать обычный файл, директорию или файл специального назначения с помощью системного вызова mknode (2), указав 0 в dev_t.

\begin{CCode}{mknode (2)}
	int mknod(
		const char *path, 
		mode_t mode, 
		dev_t dev
	); \end{CCode}

или воспользоваться функцией mkfifo (3)

\begin{CCode}{mkfifo (3)}
	int mkfifo(
		const char *pathname, 
		mode_t mode
	); \end{CCode}


		\section{Математические операции}
		Именованные каналы во многом работают так же, как и неименованные каналы, но все же имеют несколько заметных отличий.

\begin{itemize}
	\item именованные каналы существуют в виде специального файла устройства в файловой системе;
	\item процессы различного происхождения могут разделять данные через такой канал;
	\item именованный канал остается в файловой системе для дальнейшего использования и после того, как весь ввод/вывод сделан.
\end{itemize}


Существует два способа создания именованного канала:

Создать обычный файл, директорию или файл специального назначения с помощью системного вызова mknode (2), указав 0 в dev_t.

\begin{CCode}{mknode (2)}
	int mknod(
		const char *path, 
		mode_t mode, 
		dev_t dev
	); \end{CCode}

или воспользоваться функцией mkfifo (3)

\begin{CCode}{mkfifo (3)}
	int mkfifo(
		const char *pathname, 
		mode_t mode
	); \end{CCode}

			
		\section{Условные кострукции, циклы, функции, комментарии}
		Именованные каналы во многом работают так же, как и неименованные каналы, но все же имеют несколько заметных отличий.

\begin{itemize}
	\item именованные каналы существуют в виде специального файла устройства в файловой системе;
	\item процессы различного происхождения могут разделять данные через такой канал;
	\item именованный канал остается в файловой системе для дальнейшего использования и после того, как весь ввод/вывод сделан.
\end{itemize}


Существует два способа создания именованного канала:

Создать обычный файл, директорию или файл специального назначения с помощью системного вызова mknode (2), указав 0 в dev_t.

\begin{CCode}{mknode (2)}
	int mknod(
		const char *path, 
		mode_t mode, 
		dev_t dev
	); \end{CCode}

или воспользоваться функцией mkfifo (3)

\begin{CCode}{mkfifo (3)}
	int mkfifo(
		const char *pathname, 
		mode_t mode
	); \end{CCode}

		
			\subsection{Условные кострукции}
			Именованные каналы во многом работают так же, как и неименованные каналы, но все же имеют несколько заметных отличий.

\begin{itemize}
	\item именованные каналы существуют в виде специального файла устройства в файловой системе;
	\item процессы различного происхождения могут разделять данные через такой канал;
	\item именованный канал остается в файловой системе для дальнейшего использования и после того, как весь ввод/вывод сделан.
\end{itemize}


Существует два способа создания именованного канала:

Создать обычный файл, директорию или файл специального назначения с помощью системного вызова mknode (2), указав 0 в dev_t.

\begin{CCode}{mknode (2)}
	int mknod(
		const char *path, 
		mode_t mode, 
		dev_t dev
	); \end{CCode}

или воспользоваться функцией mkfifo (3)

\begin{CCode}{mkfifo (3)}
	int mkfifo(
		const char *pathname, 
		mode_t mode
	); \end{CCode}

			
			\subsection{Циклы}
			Именованные каналы во многом работают так же, как и неименованные каналы, но все же имеют несколько заметных отличий.

\begin{itemize}
	\item именованные каналы существуют в виде специального файла устройства в файловой системе;
	\item процессы различного происхождения могут разделять данные через такой канал;
	\item именованный канал остается в файловой системе для дальнейшего использования и после того, как весь ввод/вывод сделан.
\end{itemize}


Существует два способа создания именованного канала:

Создать обычный файл, директорию или файл специального назначения с помощью системного вызова mknode (2), указав 0 в dev_t.

\begin{CCode}{mknode (2)}
	int mknod(
		const char *path, 
		mode_t mode, 
		dev_t dev
	); \end{CCode}

или воспользоваться функцией mkfifo (3)

\begin{CCode}{mkfifo (3)}
	int mkfifo(
		const char *pathname, 
		mode_t mode
	); \end{CCode}

			
			\subsection{Функции}
			Именованные каналы во многом работают так же, как и неименованные каналы, но все же имеют несколько заметных отличий.

\begin{itemize}
	\item именованные каналы существуют в виде специального файла устройства в файловой системе;
	\item процессы различного происхождения могут разделять данные через такой канал;
	\item именованный канал остается в файловой системе для дальнейшего использования и после того, как весь ввод/вывод сделан.
\end{itemize}


Существует два способа создания именованного канала:

Создать обычный файл, директорию или файл специального назначения с помощью системного вызова mknode (2), указав 0 в dev_t.

\begin{CCode}{mknode (2)}
	int mknod(
		const char *path, 
		mode_t mode, 
		dev_t dev
	); \end{CCode}

или воспользоваться функцией mkfifo (3)

\begin{CCode}{mkfifo (3)}
	int mkfifo(
		const char *pathname, 
		mode_t mode
	); \end{CCode}

			
			\subsection{Комментарии}
			Именованные каналы во многом работают так же, как и неименованные каналы, но все же имеют несколько заметных отличий.

\begin{itemize}
	\item именованные каналы существуют в виде специального файла устройства в файловой системе;
	\item процессы различного происхождения могут разделять данные через такой канал;
	\item именованный канал остается в файловой системе для дальнейшего использования и после того, как весь ввод/вывод сделан.
\end{itemize}


Существует два способа создания именованного канала:

Создать обычный файл, директорию или файл специального назначения с помощью системного вызова mknode (2), указав 0 в dev_t.

\begin{CCode}{mknode (2)}
	int mknod(
		const char *path, 
		mode_t mode, 
		dev_t dev
	); \end{CCode}

или воспользоваться функцией mkfifo (3)

\begin{CCode}{mkfifo (3)}
	int mkfifo(
		const char *pathname, 
		mode_t mode
	); \end{CCode}

			
		\chapter{Потоки ввода-вывода}
		Именованные каналы во многом работают так же, как и неименованные каналы, но все же имеют несколько заметных отличий.

\begin{itemize}
	\item именованные каналы существуют в виде специального файла устройства в файловой системе;
	\item процессы различного происхождения могут разделять данные через такой канал;
	\item именованный канал остается в файловой системе для дальнейшего использования и после того, как весь ввод/вывод сделан.
\end{itemize}


Существует два способа создания именованного канала:

Создать обычный файл, директорию или файл специального назначения с помощью системного вызова mknode (2), указав 0 в dev_t.

\begin{CCode}{mknode (2)}
	int mknod(
		const char *path, 
		mode_t mode, 
		dev_t dev
	); \end{CCode}

или воспользоваться функцией mkfifo (3)

\begin{CCode}{mkfifo (3)}
	int mkfifo(
		const char *pathname, 
		mode_t mode
	); \end{CCode}

		
			\section{Стандартные потоки ввода-вывода}
			Именованные каналы во многом работают так же, как и неименованные каналы, но все же имеют несколько заметных отличий.

\begin{itemize}
	\item именованные каналы существуют в виде специального файла устройства в файловой системе;
	\item процессы различного происхождения могут разделять данные через такой канал;
	\item именованный канал остается в файловой системе для дальнейшего использования и после того, как весь ввод/вывод сделан.
\end{itemize}


Существует два способа создания именованного канала:

Создать обычный файл, директорию или файл специального назначения с помощью системного вызова mknode (2), указав 0 в dev_t.

\begin{CCode}{mknode (2)}
	int mknod(
		const char *path, 
		mode_t mode, 
		dev_t dev
	); \end{CCode}

или воспользоваться функцией mkfifo (3)

\begin{CCode}{mkfifo (3)}
	int mkfifo(
		const char *pathname, 
		mode_t mode
	); \end{CCode}

			
			\section{Перенаправление потоков}
			Именованные каналы во многом работают так же, как и неименованные каналы, но все же имеют несколько заметных отличий.

\begin{itemize}
	\item именованные каналы существуют в виде специального файла устройства в файловой системе;
	\item процессы различного происхождения могут разделять данные через такой канал;
	\item именованный канал остается в файловой системе для дальнейшего использования и после того, как весь ввод/вывод сделан.
\end{itemize}


Существует два способа создания именованного канала:

Создать обычный файл, директорию или файл специального назначения с помощью системного вызова mknode (2), указав 0 в dev_t.

\begin{CCode}{mknode (2)}
	int mknod(
		const char *path, 
		mode_t mode, 
		dev_t dev
	); \end{CCode}

или воспользоваться функцией mkfifo (3)

\begin{CCode}{mkfifo (3)}
	int mkfifo(
		const char *pathname, 
		mode_t mode
	); \end{CCode}

			
		\chapter{set и env}
		Именованные каналы во многом работают так же, как и неименованные каналы, но все же имеют несколько заметных отличий.

\begin{itemize}
	\item именованные каналы существуют в виде специального файла устройства в файловой системе;
	\item процессы различного происхождения могут разделять данные через такой канал;
	\item именованный канал остается в файловой системе для дальнейшего использования и после того, как весь ввод/вывод сделан.
\end{itemize}


Существует два способа создания именованного канала:

Создать обычный файл, директорию или файл специального назначения с помощью системного вызова mknode (2), указав 0 в dev_t.

\begin{CCode}{mknode (2)}
	int mknod(
		const char *path, 
		mode_t mode, 
		dev_t dev
	); \end{CCode}

или воспользоваться функцией mkfifo (3)

\begin{CCode}{mkfifo (3)}
	int mkfifo(
		const char *pathname, 
		mode_t mode
	); \end{CCode}

		
			\section{Команда set}
			Именованные каналы во многом работают так же, как и неименованные каналы, но все же имеют несколько заметных отличий.

\begin{itemize}
	\item именованные каналы существуют в виде специального файла устройства в файловой системе;
	\item процессы различного происхождения могут разделять данные через такой канал;
	\item именованный канал остается в файловой системе для дальнейшего использования и после того, как весь ввод/вывод сделан.
\end{itemize}


Существует два способа создания именованного канала:

Создать обычный файл, директорию или файл специального назначения с помощью системного вызова mknode (2), указав 0 в dev_t.

\begin{CCode}{mknode (2)}
	int mknod(
		const char *path, 
		mode_t mode, 
		dev_t dev
	); \end{CCode}

или воспользоваться функцией mkfifo (3)

\begin{CCode}{mkfifo (3)}
	int mkfifo(
		const char *pathname, 
		mode_t mode
	); \end{CCode}

		
			\section{Утилита env}
			Именованные каналы во многом работают так же, как и неименованные каналы, но все же имеют несколько заметных отличий.

\begin{itemize}
	\item именованные каналы существуют в виде специального файла устройства в файловой системе;
	\item процессы различного происхождения могут разделять данные через такой канал;
	\item именованный канал остается в файловой системе для дальнейшего использования и после того, как весь ввод/вывод сделан.
\end{itemize}


Существует два способа создания именованного канала:

Создать обычный файл, директорию или файл специального назначения с помощью системного вызова mknode (2), указав 0 в dev_t.

\begin{CCode}{mknode (2)}
	int mknod(
		const char *path, 
		mode_t mode, 
		dev_t dev
	); \end{CCode}

или воспользоваться функцией mkfifo (3)

\begin{CCode}{mkfifo (3)}
	int mkfifo(
		const char *pathname, 
		mode_t mode
	); \end{CCode}

			
			
			
% ------------------------------
% Часть: Язык Си в системном программировании

\part{Язык Си в системном программировании}
%Именованные каналы во многом работают так же, как и неименованные каналы, но все же имеют несколько заметных отличий.

\begin{itemize}
	\item именованные каналы существуют в виде специального файла устройства в файловой системе;
	\item процессы различного происхождения могут разделять данные через такой канал;
	\item именованный канал остается в файловой системе для дальнейшего использования и после того, как весь ввод/вывод сделан.
\end{itemize}


Существует два способа создания именованного канала:

Создать обычный файл, директорию или файл специального назначения с помощью системного вызова mknode (2), указав 0 в dev_t.

\begin{CCode}{mknode (2)}
	int mknod(
		const char *path, 
		mode_t mode, 
		dev_t dev
	); \end{CCode}

или воспользоваться функцией mkfifo (3)

\begin{CCode}{mkfifo (3)}
	int mkfifo(
		const char *pathname, 
		mode_t mode
	); \end{CCode}


	\chapter{Программы на языке Си}
	Именованные каналы во многом работают так же, как и неименованные каналы, но все же имеют несколько заметных отличий.

\begin{itemize}
	\item именованные каналы существуют в виде специального файла устройства в файловой системе;
	\item процессы различного происхождения могут разделять данные через такой канал;
	\item именованный канал остается в файловой системе для дальнейшего использования и после того, как весь ввод/вывод сделан.
\end{itemize}


Существует два способа создания именованного канала:

Создать обычный файл, директорию или файл специального назначения с помощью системного вызова mknode (2), указав 0 в dev_t.

\begin{CCode}{mknode (2)}
	int mknod(
		const char *path, 
		mode_t mode, 
		dev_t dev
	); \end{CCode}

или воспользоваться функцией mkfifo (3)

\begin{CCode}{mkfifo (3)}
	int mkfifo(
		const char *pathname, 
		mode_t mode
	); \end{CCode}

	
		\section{Заголовочные файлы}
		Именованные каналы во многом работают так же, как и неименованные каналы, но все же имеют несколько заметных отличий.

\begin{itemize}
	\item именованные каналы существуют в виде специального файла устройства в файловой системе;
	\item процессы различного происхождения могут разделять данные через такой канал;
	\item именованный канал остается в файловой системе для дальнейшего использования и после того, как весь ввод/вывод сделан.
\end{itemize}


Существует два способа создания именованного канала:

Создать обычный файл, директорию или файл специального назначения с помощью системного вызова mknode (2), указав 0 в dev_t.

\begin{CCode}{mknode (2)}
	int mknod(
		const char *path, 
		mode_t mode, 
		dev_t dev
	); \end{CCode}

или воспользоваться функцией mkfifo (3)

\begin{CCode}{mkfifo (3)}
	int mkfifo(
		const char *pathname, 
		mode_t mode
	); \end{CCode}

		
			\subsection{Часто используемые заголовочные файлы UNIX}
			Именованные каналы во многом работают так же, как и неименованные каналы, но все же имеют несколько заметных отличий.

\begin{itemize}
	\item именованные каналы существуют в виде специального файла устройства в файловой системе;
	\item процессы различного происхождения могут разделять данные через такой канал;
	\item именованный канал остается в файловой системе для дальнейшего использования и после того, как весь ввод/вывод сделан.
\end{itemize}


Существует два способа создания именованного канала:

Создать обычный файл, директорию или файл специального назначения с помощью системного вызова mknode (2), указав 0 в dev_t.

\begin{CCode}{mknode (2)}
	int mknod(
		const char *path, 
		mode_t mode, 
		dev_t dev
	); \end{CCode}

или воспользоваться функцией mkfifo (3)

\begin{CCode}{mkfifo (3)}
	int mkfifo(
		const char *pathname, 
		mode_t mode
	); \end{CCode}

	
		\section{main-функция}
		Именованные каналы во многом работают так же, как и неименованные каналы, но все же имеют несколько заметных отличий.

\begin{itemize}
	\item именованные каналы существуют в виде специального файла устройства в файловой системе;
	\item процессы различного происхождения могут разделять данные через такой канал;
	\item именованный канал остается в файловой системе для дальнейшего использования и после того, как весь ввод/вывод сделан.
\end{itemize}


Существует два способа создания именованного канала:

Создать обычный файл, директорию или файл специального назначения с помощью системного вызова mknode (2), указав 0 в dev_t.

\begin{CCode}{mknode (2)}
	int mknod(
		const char *path, 
		mode_t mode, 
		dev_t dev
	); \end{CCode}

или воспользоваться функцией mkfifo (3)

\begin{CCode}{mkfifo (3)}
	int mkfifo(
		const char *pathname, 
		mode_t mode
	); \end{CCode}

		
			\subsection{Аргументы main-функции}
			Именованные каналы во многом работают так же, как и неименованные каналы, но все же имеют несколько заметных отличий.

\begin{itemize}
	\item именованные каналы существуют в виде специального файла устройства в файловой системе;
	\item процессы различного происхождения могут разделять данные через такой канал;
	\item именованный канал остается в файловой системе для дальнейшего использования и после того, как весь ввод/вывод сделан.
\end{itemize}


Существует два способа создания именованного канала:

Создать обычный файл, директорию или файл специального назначения с помощью системного вызова mknode (2), указав 0 в dev_t.

\begin{CCode}{mknode (2)}
	int mknod(
		const char *path, 
		mode_t mode, 
		dev_t dev
	); \end{CCode}

или воспользоваться функцией mkfifo (3)

\begin{CCode}{mkfifo (3)}
	int mkfifo(
		const char *pathname, 
		mode_t mode
	); \end{CCode}

		
		\section{Код возврата}
		Именованные каналы во многом работают так же, как и неименованные каналы, но все же имеют несколько заметных отличий.

\begin{itemize}
	\item именованные каналы существуют в виде специального файла устройства в файловой системе;
	\item процессы различного происхождения могут разделять данные через такой канал;
	\item именованный канал остается в файловой системе для дальнейшего использования и после того, как весь ввод/вывод сделан.
\end{itemize}


Существует два способа создания именованного канала:

Создать обычный файл, директорию или файл специального назначения с помощью системного вызова mknode (2), указав 0 в dev_t.

\begin{CCode}{mknode (2)}
	int mknod(
		const char *path, 
		mode_t mode, 
		dev_t dev
	); \end{CCode}

или воспользоваться функцией mkfifo (3)

\begin{CCode}{mkfifo (3)}
	int mkfifo(
		const char *pathname, 
		mode_t mode
	); \end{CCode}

		
		\section{Ошибки и стандартизация ошибок}
		Именованные каналы во многом работают так же, как и неименованные каналы, но все же имеют несколько заметных отличий.

\begin{itemize}
	\item именованные каналы существуют в виде специального файла устройства в файловой системе;
	\item процессы различного происхождения могут разделять данные через такой канал;
	\item именованный канал остается в файловой системе для дальнейшего использования и после того, как весь ввод/вывод сделан.
\end{itemize}


Существует два способа создания именованного канала:

Создать обычный файл, директорию или файл специального назначения с помощью системного вызова mknode (2), указав 0 в dev_t.

\begin{CCode}{mknode (2)}
	int mknod(
		const char *path, 
		mode_t mode, 
		dev_t dev
	); \end{CCode}

или воспользоваться функцией mkfifo (3)

\begin{CCode}{mkfifo (3)}
	int mkfifo(
		const char *pathname, 
		mode_t mode
	); \end{CCode}

		
			\subsection{perror (3) и strerror (3)}
			Именованные каналы во многом работают так же, как и неименованные каналы, но все же имеют несколько заметных отличий.

\begin{itemize}
	\item именованные каналы существуют в виде специального файла устройства в файловой системе;
	\item процессы различного происхождения могут разделять данные через такой канал;
	\item именованный канал остается в файловой системе для дальнейшего использования и после того, как весь ввод/вывод сделан.
\end{itemize}


Существует два способа создания именованного канала:

Создать обычный файл, директорию или файл специального назначения с помощью системного вызова mknode (2), указав 0 в dev_t.

\begin{CCode}{mknode (2)}
	int mknod(
		const char *path, 
		mode_t mode, 
		dev_t dev
	); \end{CCode}

или воспользоваться функцией mkfifo (3)

\begin{CCode}{mkfifo (3)}
	int mkfifo(
		const char *pathname, 
		mode_t mode
	); \end{CCode}

		
	\chapter{Компиляция программы}
	Именованные каналы во многом работают так же, как и неименованные каналы, но все же имеют несколько заметных отличий.

\begin{itemize}
	\item именованные каналы существуют в виде специального файла устройства в файловой системе;
	\item процессы различного происхождения могут разделять данные через такой канал;
	\item именованный канал остается в файловой системе для дальнейшего использования и после того, как весь ввод/вывод сделан.
\end{itemize}


Существует два способа создания именованного канала:

Создать обычный файл, директорию или файл специального назначения с помощью системного вызова mknode (2), указав 0 в dev_t.

\begin{CCode}{mknode (2)}
	int mknod(
		const char *path, 
		mode_t mode, 
		dev_t dev
	); \end{CCode}

или воспользоваться функцией mkfifo (3)

\begin{CCode}{mkfifo (3)}
	int mkfifo(
		const char *pathname, 
		mode_t mode
	); \end{CCode}

	
		\section{Компиляторы}
		Именованные каналы во многом работают так же, как и неименованные каналы, но все же имеют несколько заметных отличий.

\begin{itemize}
	\item именованные каналы существуют в виде специального файла устройства в файловой системе;
	\item процессы различного происхождения могут разделять данные через такой канал;
	\item именованный канал остается в файловой системе для дальнейшего использования и после того, как весь ввод/вывод сделан.
\end{itemize}


Существует два способа создания именованного канала:

Создать обычный файл, директорию или файл специального назначения с помощью системного вызова mknode (2), указав 0 в dev_t.

\begin{CCode}{mknode (2)}
	int mknod(
		const char *path, 
		mode_t mode, 
		dev_t dev
	); \end{CCode}

или воспользоваться функцией mkfifo (3)

\begin{CCode}{mkfifo (3)}
	int mkfifo(
		const char *pathname, 
		mode_t mode
	); \end{CCode}

	
		\section{Makefile и утилита make}
		Именованные каналы во многом работают так же, как и неименованные каналы, но все же имеют несколько заметных отличий.

\begin{itemize}
	\item именованные каналы существуют в виде специального файла устройства в файловой системе;
	\item процессы различного происхождения могут разделять данные через такой канал;
	\item именованный канал остается в файловой системе для дальнейшего использования и после того, как весь ввод/вывод сделан.
\end{itemize}


Существует два способа создания именованного канала:

Создать обычный файл, директорию или файл специального назначения с помощью системного вызова mknode (2), указав 0 в dev_t.

\begin{CCode}{mknode (2)}
	int mknod(
		const char *path, 
		mode_t mode, 
		dev_t dev
	); \end{CCode}

или воспользоваться функцией mkfifo (3)

\begin{CCode}{mkfifo (3)}
	int mkfifo(
		const char *pathname, 
		mode_t mode
	); \end{CCode}

	
	
% ------------------------------
% Часть: Файлы и файловая система

\part{Файлы и файловая подсистема}
%Именованные каналы во многом работают так же, как и неименованные каналы, но все же имеют несколько заметных отличий.

\begin{itemize}
	\item именованные каналы существуют в виде специального файла устройства в файловой системе;
	\item процессы различного происхождения могут разделять данные через такой канал;
	\item именованный канал остается в файловой системе для дальнейшего использования и после того, как весь ввод/вывод сделан.
\end{itemize}


Существует два способа создания именованного канала:

Создать обычный файл, директорию или файл специального назначения с помощью системного вызова mknode (2), указав 0 в dev_t.

\begin{CCode}{mknode (2)}
	int mknod(
		const char *path, 
		mode_t mode, 
		dev_t dev
	); \end{CCode}

или воспользоваться функцией mkfifo (3)

\begin{CCode}{mkfifo (3)}
	int mkfifo(
		const char *pathname, 
		mode_t mode
	); \end{CCode}

	
	\chapter{Все есть файл! (Кроме потоков и ядра)}
	Именованные каналы во многом работают так же, как и неименованные каналы, но все же имеют несколько заметных отличий.

\begin{itemize}
	\item именованные каналы существуют в виде специального файла устройства в файловой системе;
	\item процессы различного происхождения могут разделять данные через такой канал;
	\item именованный канал остается в файловой системе для дальнейшего использования и после того, как весь ввод/вывод сделан.
\end{itemize}


Существует два способа создания именованного канала:

Создать обычный файл, директорию или файл специального назначения с помощью системного вызова mknode (2), указав 0 в dev_t.

\begin{CCode}{mknode (2)}
	int mknod(
		const char *path, 
		mode_t mode, 
		dev_t dev
	); \end{CCode}

или воспользоваться функцией mkfifo (3)

\begin{CCode}{mkfifo (3)}
	int mkfifo(
		const char *pathname, 
		mode_t mode
	); \end{CCode}

	
		\section{inode}
		Именованные каналы во многом работают так же, как и неименованные каналы, но все же имеют несколько заметных отличий.

\begin{itemize}
	\item именованные каналы существуют в виде специального файла устройства в файловой системе;
	\item процессы различного происхождения могут разделять данные через такой канал;
	\item именованный канал остается в файловой системе для дальнейшего использования и после того, как весь ввод/вывод сделан.
\end{itemize}


Существует два способа создания именованного канала:

Создать обычный файл, директорию или файл специального назначения с помощью системного вызова mknode (2), указав 0 в dev_t.

\begin{CCode}{mknode (2)}
	int mknod(
		const char *path, 
		mode_t mode, 
		dev_t dev
	); \end{CCode}

или воспользоваться функцией mkfifo (3)

\begin{CCode}{mkfifo (3)}
	int mkfifo(
		const char *pathname, 
		mode_t mode
	); \end{CCode}

		
			\section{Структура stat}
			Именованные каналы во многом работают так же, как и неименованные каналы, но все же имеют несколько заметных отличий.

\begin{itemize}
	\item именованные каналы существуют в виде специального файла устройства в файловой системе;
	\item процессы различного происхождения могут разделять данные через такой канал;
	\item именованный канал остается в файловой системе для дальнейшего использования и после того, как весь ввод/вывод сделан.
\end{itemize}


Существует два способа создания именованного канала:

Создать обычный файл, директорию или файл специального назначения с помощью системного вызова mknode (2), указав 0 в dev_t.

\begin{CCode}{mknode (2)}
	int mknod(
		const char *path, 
		mode_t mode, 
		dev_t dev
	); \end{CCode}

или воспользоваться функцией mkfifo (3)

\begin{CCode}{mkfifo (3)}
	int mkfifo(
		const char *pathname, 
		mode_t mode
	); \end{CCode}

			
		\section{Файловый дескриптор}
		Именованные каналы во многом работают так же, как и неименованные каналы, но все же имеют несколько заметных отличий.

\begin{itemize}
	\item именованные каналы существуют в виде специального файла устройства в файловой системе;
	\item процессы различного происхождения могут разделять данные через такой канал;
	\item именованный канал остается в файловой системе для дальнейшего использования и после того, как весь ввод/вывод сделан.
\end{itemize}


Существует два способа создания именованного канала:

Создать обычный файл, директорию или файл специального назначения с помощью системного вызова mknode (2), указав 0 в dev_t.

\begin{CCode}{mknode (2)}
	int mknod(
		const char *path, 
		mode_t mode, 
		dev_t dev
	); \end{CCode}

или воспользоваться функцией mkfifo (3)

\begin{CCode}{mkfifo (3)}
	int mkfifo(
		const char *pathname, 
		mode_t mode
	); \end{CCode}

			
			\subsection{Работа с файловым дескриптором}
			Именованные каналы во многом работают так же, как и неименованные каналы, но все же имеют несколько заметных отличий.

\begin{itemize}
	\item именованные каналы существуют в виде специального файла устройства в файловой системе;
	\item процессы различного происхождения могут разделять данные через такой канал;
	\item именованный канал остается в файловой системе для дальнейшего использования и после того, как весь ввод/вывод сделан.
\end{itemize}


Существует два способа создания именованного канала:

Создать обычный файл, директорию или файл специального назначения с помощью системного вызова mknode (2), указав 0 в dev_t.

\begin{CCode}{mknode (2)}
	int mknod(
		const char *path, 
		mode_t mode, 
		dev_t dev
	); \end{CCode}

или воспользоваться функцией mkfifo (3)

\begin{CCode}{mkfifo (3)}
	int mkfifo(
		const char *pathname, 
		mode_t mode
	); \end{CCode}

			
			\subsection{fcntl}
			Именованные каналы во многом работают так же, как и неименованные каналы, но все же имеют несколько заметных отличий.

\begin{itemize}
	\item именованные каналы существуют в виде специального файла устройства в файловой системе;
	\item процессы различного происхождения могут разделять данные через такой канал;
	\item именованный канал остается в файловой системе для дальнейшего использования и после того, как весь ввод/вывод сделан.
\end{itemize}


Существует два способа создания именованного канала:

Создать обычный файл, директорию или файл специального назначения с помощью системного вызова mknode (2), указав 0 в dev_t.

\begin{CCode}{mknode (2)}
	int mknod(
		const char *path, 
		mode_t mode, 
		dev_t dev
	); \end{CCode}

или воспользоваться функцией mkfifo (3)

\begin{CCode}{mkfifo (3)}
	int mkfifo(
		const char *pathname, 
		mode_t mode
	); \end{CCode}

	
		\section{Типы файлов}
		Именованные каналы во многом работают так же, как и неименованные каналы, но все же имеют несколько заметных отличий.

\begin{itemize}
	\item именованные каналы существуют в виде специального файла устройства в файловой системе;
	\item процессы различного происхождения могут разделять данные через такой канал;
	\item именованный канал остается в файловой системе для дальнейшего использования и после того, как весь ввод/вывод сделан.
\end{itemize}


Существует два способа создания именованного канала:

Создать обычный файл, директорию или файл специального назначения с помощью системного вызова mknode (2), указав 0 в dev_t.

\begin{CCode}{mknode (2)}
	int mknod(
		const char *path, 
		mode_t mode, 
		dev_t dev
	); \end{CCode}

или воспользоваться функцией mkfifo (3)

\begin{CCode}{mkfifo (3)}
	int mkfifo(
		const char *pathname, 
		mode_t mode
	); \end{CCode}

		
	\chapter{Файловый ввод-вывод}
	Именованные каналы во многом работают так же, как и неименованные каналы, но все же имеют несколько заметных отличий.

\begin{itemize}
	\item именованные каналы существуют в виде специального файла устройства в файловой системе;
	\item процессы различного происхождения могут разделять данные через такой канал;
	\item именованный канал остается в файловой системе для дальнейшего использования и после того, как весь ввод/вывод сделан.
\end{itemize}


Существует два способа создания именованного канала:

Создать обычный файл, директорию или файл специального назначения с помощью системного вызова mknode (2), указав 0 в dev_t.

\begin{CCode}{mknode (2)}
	int mknod(
		const char *path, 
		mode_t mode, 
		dev_t dev
	); \end{CCode}

или воспользоваться функцией mkfifo (3)

\begin{CCode}{mkfifo (3)}
	int mkfifo(
		const char *pathname, 
		mode_t mode
	); \end{CCode}

	
		\section{Открытие и закрытие файла}
		Именованные каналы во многом работают так же, как и неименованные каналы, но все же имеют несколько заметных отличий.

\begin{itemize}
	\item именованные каналы существуют в виде специального файла устройства в файловой системе;
	\item процессы различного происхождения могут разделять данные через такой канал;
	\item именованный канал остается в файловой системе для дальнейшего использования и после того, как весь ввод/вывод сделан.
\end{itemize}


Существует два способа создания именованного канала:

Создать обычный файл, директорию или файл специального назначения с помощью системного вызова mknode (2), указав 0 в dev_t.

\begin{CCode}{mknode (2)}
	int mknod(
		const char *path, 
		mode_t mode, 
		dev_t dev
	); \end{CCode}

или воспользоваться функцией mkfifo (3)

\begin{CCode}{mkfifo (3)}
	int mkfifo(
		const char *pathname, 
		mode_t mode
	); \end{CCode}

		
		\section{Перемещение внутри файла}
		Именованные каналы во многом работают так же, как и неименованные каналы, но все же имеют несколько заметных отличий.

\begin{itemize}
	\item именованные каналы существуют в виде специального файла устройства в файловой системе;
	\item процессы различного происхождения могут разделять данные через такой канал;
	\item именованный канал остается в файловой системе для дальнейшего использования и после того, как весь ввод/вывод сделан.
\end{itemize}


Существует два способа создания именованного канала:

Создать обычный файл, директорию или файл специального назначения с помощью системного вызова mknode (2), указав 0 в dev_t.

\begin{CCode}{mknode (2)}
	int mknod(
		const char *path, 
		mode_t mode, 
		dev_t dev
	); \end{CCode}

или воспользоваться функцией mkfifo (3)

\begin{CCode}{mkfifo (3)}
	int mkfifo(
		const char *pathname, 
		mode_t mode
	); \end{CCode}

		
		\section{Ввод-вывод}
		Именованные каналы во многом работают так же, как и неименованные каналы, но все же имеют несколько заметных отличий.

\begin{itemize}
	\item именованные каналы существуют в виде специального файла устройства в файловой системе;
	\item процессы различного происхождения могут разделять данные через такой канал;
	\item именованный канал остается в файловой системе для дальнейшего использования и после того, как весь ввод/вывод сделан.
\end{itemize}


Существует два способа создания именованного канала:

Создать обычный файл, директорию или файл специального назначения с помощью системного вызова mknode (2), указав 0 в dev_t.

\begin{CCode}{mknode (2)}
	int mknod(
		const char *path, 
		mode_t mode, 
		dev_t dev
	); \end{CCode}

или воспользоваться функцией mkfifo (3)

\begin{CCode}{mkfifo (3)}
	int mkfifo(
		const char *pathname, 
		mode_t mode
	); \end{CCode}

	
			\subsection{Простой ввод-вывод}
			Именованные каналы во многом работают так же, как и неименованные каналы, но все же имеют несколько заметных отличий.

\begin{itemize}
	\item именованные каналы существуют в виде специального файла устройства в файловой системе;
	\item процессы различного происхождения могут разделять данные через такой канал;
	\item именованный канал остается в файловой системе для дальнейшего использования и после того, как весь ввод/вывод сделан.
\end{itemize}


Существует два способа создания именованного канала:

Создать обычный файл, директорию или файл специального назначения с помощью системного вызова mknode (2), указав 0 в dev_t.

\begin{CCode}{mknode (2)}
	int mknod(
		const char *path, 
		mode_t mode, 
		dev_t dev
	); \end{CCode}

или воспользоваться функцией mkfifo (3)

\begin{CCode}{mkfifo (3)}
	int mkfifo(
		const char *pathname, 
		mode_t mode
	); \end{CCode}

			
			\subsection{Ввод-вывод со смещением}
			Именованные каналы во многом работают так же, как и неименованные каналы, но все же имеют несколько заметных отличий.

\begin{itemize}
	\item именованные каналы существуют в виде специального файла устройства в файловой системе;
	\item процессы различного происхождения могут разделять данные через такой канал;
	\item именованный канал остается в файловой системе для дальнейшего использования и после того, как весь ввод/вывод сделан.
\end{itemize}


Существует два способа создания именованного канала:

Создать обычный файл, директорию или файл специального назначения с помощью системного вызова mknode (2), указав 0 в dev_t.

\begin{CCode}{mknode (2)}
	int mknod(
		const char *path, 
		mode_t mode, 
		dev_t dev
	); \end{CCode}

или воспользоваться функцией mkfifo (3)

\begin{CCode}{mkfifo (3)}
	int mkfifo(
		const char *pathname, 
		mode_t mode
	); \end{CCode}

			
			\subsection{Векторный ввод-вывод}
			Именованные каналы во многом работают так же, как и неименованные каналы, но все же имеют несколько заметных отличий.

\begin{itemize}
	\item именованные каналы существуют в виде специального файла устройства в файловой системе;
	\item процессы различного происхождения могут разделять данные через такой канал;
	\item именованный канал остается в файловой системе для дальнейшего использования и после того, как весь ввод/вывод сделан.
\end{itemize}


Существует два способа создания именованного канала:

Создать обычный файл, директорию или файл специального назначения с помощью системного вызова mknode (2), указав 0 в dev_t.

\begin{CCode}{mknode (2)}
	int mknod(
		const char *path, 
		mode_t mode, 
		dev_t dev
	); \end{CCode}

или воспользоваться функцией mkfifo (3)

\begin{CCode}{mkfifo (3)}
	int mkfifo(
		const char *pathname, 
		mode_t mode
	); \end{CCode}

			
		\subsection{Конец файла}
		Именованные каналы во многом работают так же, как и неименованные каналы, но все же имеют несколько заметных отличий.

\begin{itemize}
	\item именованные каналы существуют в виде специального файла устройства в файловой системе;
	\item процессы различного происхождения могут разделять данные через такой канал;
	\item именованный канал остается в файловой системе для дальнейшего использования и после того, как весь ввод/вывод сделан.
\end{itemize}


Существует два способа создания именованного канала:

Создать обычный файл, директорию или файл специального назначения с помощью системного вызова mknode (2), указав 0 в dev_t.

\begin{CCode}{mknode (2)}
	int mknod(
		const char *path, 
		mode_t mode, 
		dev_t dev
	); \end{CCode}

или воспользоваться функцией mkfifo (3)

\begin{CCode}{mkfifo (3)}
	int mkfifo(
		const char *pathname, 
		mode_t mode
	); \end{CCode}

		
	\chapter{Директории}
	Именованные каналы во многом работают так же, как и неименованные каналы, но все же имеют несколько заметных отличий.

\begin{itemize}
	\item именованные каналы существуют в виде специального файла устройства в файловой системе;
	\item процессы различного происхождения могут разделять данные через такой канал;
	\item именованный канал остается в файловой системе для дальнейшего использования и после того, как весь ввод/вывод сделан.
\end{itemize}


Существует два способа создания именованного канала:

Создать обычный файл, директорию или файл специального назначения с помощью системного вызова mknode (2), указав 0 в dev_t.

\begin{CCode}{mknode (2)}
	int mknod(
		const char *path, 
		mode_t mode, 
		dev_t dev
	); \end{CCode}

или воспользоваться функцией mkfifo (3)

\begin{CCode}{mkfifo (3)}
	int mkfifo(
		const char *pathname, 
		mode_t mode
	); \end{CCode}
	
	
		\section{Структура dirent}
		Именованные каналы во многом работают так же, как и неименованные каналы, но все же имеют несколько заметных отличий.

\begin{itemize}
	\item именованные каналы существуют в виде специального файла устройства в файловой системе;
	\item процессы различного происхождения могут разделять данные через такой канал;
	\item именованный канал остается в файловой системе для дальнейшего использования и после того, как весь ввод/вывод сделан.
\end{itemize}


Существует два способа создания именованного канала:

Создать обычный файл, директорию или файл специального назначения с помощью системного вызова mknode (2), указав 0 в dev_t.

\begin{CCode}{mknode (2)}
	int mknod(
		const char *path, 
		mode_t mode, 
		dev_t dev
	); \end{CCode}

или воспользоваться функцией mkfifo (3)

\begin{CCode}{mkfifo (3)}
	int mkfifo(
		const char *pathname, 
		mode_t mode
	); \end{CCode}

		
		\section{Получение текущей рабочей директории и ее изменение}
		Именованные каналы во многом работают так же, как и неименованные каналы, но все же имеют несколько заметных отличий.

\begin{itemize}
	\item именованные каналы существуют в виде специального файла устройства в файловой системе;
	\item процессы различного происхождения могут разделять данные через такой канал;
	\item именованный канал остается в файловой системе для дальнейшего использования и после того, как весь ввод/вывод сделан.
\end{itemize}


Существует два способа создания именованного канала:

Создать обычный файл, директорию или файл специального назначения с помощью системного вызова mknode (2), указав 0 в dev_t.

\begin{CCode}{mknode (2)}
	int mknod(
		const char *path, 
		mode_t mode, 
		dev_t dev
	); \end{CCode}

или воспользоваться функцией mkfifo (3)

\begin{CCode}{mkfifo (3)}
	int mkfifo(
		const char *pathname, 
		mode_t mode
	); \end{CCode}
	
		
		\section{Создание и удаление директорий}
		Именованные каналы во многом работают так же, как и неименованные каналы, но все же имеют несколько заметных отличий.

\begin{itemize}
	\item именованные каналы существуют в виде специального файла устройства в файловой системе;
	\item процессы различного происхождения могут разделять данные через такой канал;
	\item именованный канал остается в файловой системе для дальнейшего использования и после того, как весь ввод/вывод сделан.
\end{itemize}


Существует два способа создания именованного канала:

Создать обычный файл, директорию или файл специального назначения с помощью системного вызова mknode (2), указав 0 в dev_t.

\begin{CCode}{mknode (2)}
	int mknod(
		const char *path, 
		mode_t mode, 
		dev_t dev
	); \end{CCode}

или воспользоваться функцией mkfifo (3)

\begin{CCode}{mkfifo (3)}
	int mkfifo(
		const char *pathname, 
		mode_t mode
	); \end{CCode}
	
		
		
	\chapter{Ссылки}
	Именованные каналы во многом работают так же, как и неименованные каналы, но все же имеют несколько заметных отличий.

\begin{itemize}
	\item именованные каналы существуют в виде специального файла устройства в файловой системе;
	\item процессы различного происхождения могут разделять данные через такой канал;
	\item именованный канал остается в файловой системе для дальнейшего использования и после того, как весь ввод/вывод сделан.
\end{itemize}


Существует два способа создания именованного канала:

Создать обычный файл, директорию или файл специального назначения с помощью системного вызова mknode (2), указав 0 в dev_t.

\begin{CCode}{mknode (2)}
	int mknod(
		const char *path, 
		mode_t mode, 
		dev_t dev
	); \end{CCode}

или воспользоваться функцией mkfifo (3)

\begin{CCode}{mkfifo (3)}
	int mkfifo(
		const char *pathname, 
		mode_t mode
	); \end{CCode}

	
		\section{Жесткие ссылки}
		Именованные каналы во многом работают так же, как и неименованные каналы, но все же имеют несколько заметных отличий.

\begin{itemize}
	\item именованные каналы существуют в виде специального файла устройства в файловой системе;
	\item процессы различного происхождения могут разделять данные через такой канал;
	\item именованный канал остается в файловой системе для дальнейшего использования и после того, как весь ввод/вывод сделан.
\end{itemize}


Существует два способа создания именованного канала:

Создать обычный файл, директорию или файл специального назначения с помощью системного вызова mknode (2), указав 0 в dev_t.

\begin{CCode}{mknode (2)}
	int mknod(
		const char *path, 
		mode_t mode, 
		dev_t dev
	); \end{CCode}

или воспользоваться функцией mkfifo (3)

\begin{CCode}{mkfifo (3)}
	int mkfifo(
		const char *pathname, 
		mode_t mode
	); \end{CCode}

		
		\section{Символьные ссылки}
		Именованные каналы во многом работают так же, как и неименованные каналы, но все же имеют несколько заметных отличий.

\begin{itemize}
	\item именованные каналы существуют в виде специального файла устройства в файловой системе;
	\item процессы различного происхождения могут разделять данные через такой канал;
	\item именованный канал остается в файловой системе для дальнейшего использования и после того, как весь ввод/вывод сделан.
\end{itemize}


Существует два способа создания именованного канала:

Создать обычный файл, директорию или файл специального назначения с помощью системного вызова mknode (2), указав 0 в dev_t.

\begin{CCode}{mknode (2)}
	int mknod(
		const char *path, 
		mode_t mode, 
		dev_t dev
	); \end{CCode}

или воспользоваться функцией mkfifo (3)

\begin{CCode}{mkfifo (3)}
	int mkfifo(
		const char *pathname, 
		mode_t mode
	); \end{CCode}


	\chapter{Файлы устройств}
	Именованные каналы во многом работают так же, как и неименованные каналы, но все же имеют несколько заметных отличий.

\begin{itemize}
	\item именованные каналы существуют в виде специального файла устройства в файловой системе;
	\item процессы различного происхождения могут разделять данные через такой канал;
	\item именованный канал остается в файловой системе для дальнейшего использования и после того, как весь ввод/вывод сделан.
\end{itemize}


Существует два способа создания именованного канала:

Создать обычный файл, директорию или файл специального назначения с помощью системного вызова mknode (2), указав 0 в dev_t.

\begin{CCode}{mknode (2)}
	int mknod(
		const char *path, 
		mode_t mode, 
		dev_t dev
	); \end{CCode}

или воспользоваться функцией mkfifo (3)

\begin{CCode}{mkfifo (3)}
	int mkfifo(
		const char *pathname, 
		mode_t mode
	); \end{CCode}




% ------------------------------
% Часть: Пользователи и группы

\part{Пользователи и группы}
%Именованные каналы во многом работают так же, как и неименованные каналы, но все же имеют несколько заметных отличий.

\begin{itemize}
	\item именованные каналы существуют в виде специального файла устройства в файловой системе;
	\item процессы различного происхождения могут разделять данные через такой канал;
	\item именованный канал остается в файловой системе для дальнейшего использования и после того, как весь ввод/вывод сделан.
\end{itemize}


Существует два способа создания именованного канала:

Создать обычный файл, директорию или файл специального назначения с помощью системного вызова mknode (2), указав 0 в dev_t.

\begin{CCode}{mknode (2)}
	int mknod(
		const char *path, 
		mode_t mode, 
		dev_t dev
	); \end{CCode}

или воспользоваться функцией mkfifo (3)

\begin{CCode}{mkfifo (3)}
	int mkfifo(
		const char *pathname, 
		mode_t mode
	); \end{CCode}


	\chapter{Пользователи и группы}
	Именованные каналы во многом работают так же, как и неименованные каналы, но все же имеют несколько заметных отличий.

\begin{itemize}
	\item именованные каналы существуют в виде специального файла устройства в файловой системе;
	\item процессы различного происхождения могут разделять данные через такой канал;
	\item именованный канал остается в файловой системе для дальнейшего использования и после того, как весь ввод/вывод сделан.
\end{itemize}


Существует два способа создания именованного канала:

Создать обычный файл, директорию или файл специального назначения с помощью системного вызова mknode (2), указав 0 в dev_t.

\begin{CCode}{mknode (2)}
	int mknod(
		const char *path, 
		mode_t mode, 
		dev_t dev
	); \end{CCode}

или воспользоваться функцией mkfifo (3)

\begin{CCode}{mkfifo (3)}
	int mkfifo(
		const char *pathname, 
		mode_t mode
	); \end{CCode}

	
		\section{Атрибуты пользователя}
		Именованные каналы во многом работают так же, как и неименованные каналы, но все же имеют несколько заметных отличий.

\begin{itemize}
	\item именованные каналы существуют в виде специального файла устройства в файловой системе;
	\item процессы различного происхождения могут разделять данные через такой канал;
	\item именованный канал остается в файловой системе для дальнейшего использования и после того, как весь ввод/вывод сделан.
\end{itemize}


Существует два способа создания именованного канала:

Создать обычный файл, директорию или файл специального назначения с помощью системного вызова mknode (2), указав 0 в dev_t.

\begin{CCode}{mknode (2)}
	int mknod(
		const char *path, 
		mode_t mode, 
		dev_t dev
	); \end{CCode}

или воспользоваться функцией mkfifo (3)

\begin{CCode}{mkfifo (3)}
	int mkfifo(
		const char *pathname, 
		mode_t mode
	); \end{CCode}

		
			\subsection{Атрибуты группы}
			Именованные каналы во многом работают так же, как и неименованные каналы, но все же имеют несколько заметных отличий.

\begin{itemize}
	\item именованные каналы существуют в виде специального файла устройства в файловой системе;
	\item процессы различного происхождения могут разделять данные через такой канал;
	\item именованный канал остается в файловой системе для дальнейшего использования и после того, как весь ввод/вывод сделан.
\end{itemize}


Существует два способа создания именованного канала:

Создать обычный файл, директорию или файл специального назначения с помощью системного вызова mknode (2), указав 0 в dev_t.

\begin{CCode}{mknode (2)}
	int mknod(
		const char *path, 
		mode_t mode, 
		dev_t dev
	); \end{CCode}

или воспользоваться функцией mkfifo (3)

\begin{CCode}{mkfifo (3)}
	int mkfifo(
		const char *pathname, 
		mode_t mode
	); \end{CCode}

		
			\subsection{Утилита getent}
			Именованные каналы во многом работают так же, как и неименованные каналы, но все же имеют несколько заметных отличий.

\begin{itemize}
	\item именованные каналы существуют в виде специального файла устройства в файловой системе;
	\item процессы различного происхождения могут разделять данные через такой канал;
	\item именованный канал остается в файловой системе для дальнейшего использования и после того, как весь ввод/вывод сделан.
\end{itemize}


Существует два способа создания именованного канала:

Создать обычный файл, директорию или файл специального назначения с помощью системного вызова mknode (2), указав 0 в dev_t.

\begin{CCode}{mknode (2)}
	int mknod(
		const char *path, 
		mode_t mode, 
		dev_t dev
	); \end{CCode}

или воспользоваться функцией mkfifo (3)

\begin{CCode}{mkfifo (3)}
	int mkfifo(
		const char *pathname, 
		mode_t mode
	); \end{CCode}

		
		

% ------------------------------
% Часть: Права и владельцы

\part{Владельцы файлов, права и режимы доступа}
%Именованные каналы во многом работают так же, как и неименованные каналы, но все же имеют несколько заметных отличий.

\begin{itemize}
	\item именованные каналы существуют в виде специального файла устройства в файловой системе;
	\item процессы различного происхождения могут разделять данные через такой канал;
	\item именованный канал остается в файловой системе для дальнейшего использования и после того, как весь ввод/вывод сделан.
\end{itemize}


Существует два способа создания именованного канала:

Создать обычный файл, директорию или файл специального назначения с помощью системного вызова mknode (2), указав 0 в dev_t.

\begin{CCode}{mknode (2)}
	int mknod(
		const char *path, 
		mode_t mode, 
		dev_t dev
	); \end{CCode}

или воспользоваться функцией mkfifo (3)

\begin{CCode}{mkfifo (3)}
	int mkfifo(
		const char *pathname, 
		mode_t mode
	); \end{CCode}


	\chapter{Права и режимы доступа}
	Именованные каналы во многом работают так же, как и неименованные каналы, но все же имеют несколько заметных отличий.

\begin{itemize}
	\item именованные каналы существуют в виде специального файла устройства в файловой системе;
	\item процессы различного происхождения могут разделять данные через такой канал;
	\item именованный канал остается в файловой системе для дальнейшего использования и после того, как весь ввод/вывод сделан.
\end{itemize}


Существует два способа создания именованного канала:

Создать обычный файл, директорию или файл специального назначения с помощью системного вызова mknode (2), указав 0 в dev_t.

\begin{CCode}{mknode (2)}
	int mknod(
		const char *path, 
		mode_t mode, 
		dev_t dev
	); \end{CCode}

или воспользоваться функцией mkfifo (3)

\begin{CCode}{mkfifo (3)}
	int mkfifo(
		const char *pathname, 
		mode_t mode
	); \end{CCode}

	
		\section{Права доступа}
		Именованные каналы во многом работают так же, как и неименованные каналы, но все же имеют несколько заметных отличий.

\begin{itemize}
	\item именованные каналы существуют в виде специального файла устройства в файловой системе;
	\item процессы различного происхождения могут разделять данные через такой канал;
	\item именованный канал остается в файловой системе для дальнейшего использования и после того, как весь ввод/вывод сделан.
\end{itemize}


Существует два способа создания именованного канала:

Создать обычный файл, директорию или файл специального назначения с помощью системного вызова mknode (2), указав 0 в dev_t.

\begin{CCode}{mknode (2)}
	int mknod(
		const char *path, 
		mode_t mode, 
		dev_t dev
	); \end{CCode}

или воспользоваться функцией mkfifo (3)

\begin{CCode}{mkfifo (3)}
	int mkfifo(
		const char *pathname, 
		mode_t mode
	); \end{CCode}

		
			\subsection{Проверка прав доступа}
			Именованные каналы во многом работают так же, как и неименованные каналы, но все же имеют несколько заметных отличий.

\begin{itemize}
	\item именованные каналы существуют в виде специального файла устройства в файловой системе;
	\item процессы различного происхождения могут разделять данные через такой канал;
	\item именованный канал остается в файловой системе для дальнейшего использования и после того, как весь ввод/вывод сделан.
\end{itemize}


Существует два способа создания именованного канала:

Создать обычный файл, директорию или файл специального назначения с помощью системного вызова mknode (2), указав 0 в dev_t.

\begin{CCode}{mknode (2)}
	int mknod(
		const char *path, 
		mode_t mode, 
		dev_t dev
	); \end{CCode}

или воспользоваться функцией mkfifo (3)

\begin{CCode}{mkfifo (3)}
	int mkfifo(
		const char *pathname, 
		mode_t mode
	); \end{CCode}

			
			\subsection{Изменение прав доступа}
			Именованные каналы во многом работают так же, как и неименованные каналы, но все же имеют несколько заметных отличий.

\begin{itemize}
	\item именованные каналы существуют в виде специального файла устройства в файловой системе;
	\item процессы различного происхождения могут разделять данные через такой канал;
	\item именованный канал остается в файловой системе для дальнейшего использования и после того, как весь ввод/вывод сделан.
\end{itemize}


Существует два способа создания именованного канала:

Создать обычный файл, директорию или файл специального назначения с помощью системного вызова mknode (2), указав 0 в dev_t.

\begin{CCode}{mknode (2)}
	int mknod(
		const char *path, 
		mode_t mode, 
		dev_t dev
	); \end{CCode}

или воспользоваться функцией mkfifo (3)

\begin{CCode}{mkfifo (3)}
	int mkfifo(
		const char *pathname, 
		mode_t mode
	); \end{CCode}

		
			\subsection{Маска создания файла}
			Именованные каналы во многом работают так же, как и неименованные каналы, но все же имеют несколько заметных отличий.

\begin{itemize}
	\item именованные каналы существуют в виде специального файла устройства в файловой системе;
	\item процессы различного происхождения могут разделять данные через такой канал;
	\item именованный канал остается в файловой системе для дальнейшего использования и после того, как весь ввод/вывод сделан.
\end{itemize}


Существует два способа создания именованного канала:

Создать обычный файл, директорию или файл специального назначения с помощью системного вызова mknode (2), указав 0 в dev_t.

\begin{CCode}{mknode (2)}
	int mknod(
		const char *path, 
		mode_t mode, 
		dev_t dev
	); \end{CCode}

или воспользоваться функцией mkfifo (3)

\begin{CCode}{mkfifo (3)}
	int mkfifo(
		const char *pathname, 
		mode_t mode
	); \end{CCode}

			
		
	\chapter{Владельцы файлов}
	Именованные каналы во многом работают так же, как и неименованные каналы, но все же имеют несколько заметных отличий.

\begin{itemize}
	\item именованные каналы существуют в виде специального файла устройства в файловой системе;
	\item процессы различного происхождения могут разделять данные через такой канал;
	\item именованный канал остается в файловой системе для дальнейшего использования и после того, как весь ввод/вывод сделан.
\end{itemize}


Существует два способа создания именованного канала:

Создать обычный файл, директорию или файл специального назначения с помощью системного вызова mknode (2), указав 0 в dev_t.

\begin{CCode}{mknode (2)}
	int mknod(
		const char *path, 
		mode_t mode, 
		dev_t dev
	); \end{CCode}

или воспользоваться функцией mkfifo (3)

\begin{CCode}{mkfifo (3)}
	int mkfifo(
		const char *pathname, 
		mode_t mode
	); \end{CCode}

	
		\section{Изменение владельца файла}
		Именованные каналы во многом работают так же, как и неименованные каналы, но все же имеют несколько заметных отличий.

\begin{itemize}
	\item именованные каналы существуют в виде специального файла устройства в файловой системе;
	\item процессы различного происхождения могут разделять данные через такой канал;
	\item именованный канал остается в файловой системе для дальнейшего использования и после того, как весь ввод/вывод сделан.
\end{itemize}


Существует два способа создания именованного канала:

Создать обычный файл, директорию или файл специального назначения с помощью системного вызова mknode (2), указав 0 в dev_t.

\begin{CCode}{mknode (2)}
	int mknod(
		const char *path, 
		mode_t mode, 
		dev_t dev
	); \end{CCode}

или воспользоваться функцией mkfifo (3)

\begin{CCode}{mkfifo (3)}
	int mkfifo(
		const char *pathname, 
		mode_t mode
	); \end{CCode}

	

% ------------------------------
% Часть: Память

\part{Память}
%Именованные каналы во многом работают так же, как и неименованные каналы, но все же имеют несколько заметных отличий.

\begin{itemize}
	\item именованные каналы существуют в виде специального файла устройства в файловой системе;
	\item процессы различного происхождения могут разделять данные через такой канал;
	\item именованный канал остается в файловой системе для дальнейшего использования и после того, как весь ввод/вывод сделан.
\end{itemize}


Существует два способа создания именованного канала:

Создать обычный файл, директорию или файл специального назначения с помощью системного вызова mknode (2), указав 0 в dev_t.

\begin{CCode}{mknode (2)}
	int mknod(
		const char *path, 
		mode_t mode, 
		dev_t dev
	); \end{CCode}

или воспользоваться функцией mkfifo (3)

\begin{CCode}{mkfifo (3)}
	int mkfifo(
		const char *pathname, 
		mode_t mode
	); \end{CCode}


	\chapter{Как устроена память}
	Именованные каналы во многом работают так же, как и неименованные каналы, но все же имеют несколько заметных отличий.

\begin{itemize}
	\item именованные каналы существуют в виде специального файла устройства в файловой системе;
	\item процессы различного происхождения могут разделять данные через такой канал;
	\item именованный канал остается в файловой системе для дальнейшего использования и после того, как весь ввод/вывод сделан.
\end{itemize}


Существует два способа создания именованного канала:

Создать обычный файл, директорию или файл специального назначения с помощью системного вызова mknode (2), указав 0 в dev_t.

\begin{CCode}{mknode (2)}
	int mknod(
		const char *path, 
		mode_t mode, 
		dev_t dev
	); \end{CCode}

или воспользоваться функцией mkfifo (3)

\begin{CCode}{mkfifo (3)}
	int mkfifo(
		const char *pathname, 
		mode_t mode
	); \end{CCode}


		\section{Аллокация памяти}
		Именованные каналы во многом работают так же, как и неименованные каналы, но все же имеют несколько заметных отличий.

\begin{itemize}
	\item именованные каналы существуют в виде специального файла устройства в файловой системе;
	\item процессы различного происхождения могут разделять данные через такой канал;
	\item именованный канал остается в файловой системе для дальнейшего использования и после того, как весь ввод/вывод сделан.
\end{itemize}


Существует два способа создания именованного канала:

Создать обычный файл, директорию или файл специального назначения с помощью системного вызова mknode (2), указав 0 в dev_t.

\begin{CCode}{mknode (2)}
	int mknod(
		const char *path, 
		mode_t mode, 
		dev_t dev
	); \end{CCode}

или воспользоваться функцией mkfifo (3)

\begin{CCode}{mkfifo (3)}
	int mkfifo(
		const char *pathname, 
		mode_t mode
	); \end{CCode}

	
% ------------------------------
% Часть: Процессы и потоки
	
\part{Процессы и потоки}
%Именованные каналы во многом работают так же, как и неименованные каналы, но все же имеют несколько заметных отличий.

\begin{itemize}
	\item именованные каналы существуют в виде специального файла устройства в файловой системе;
	\item процессы различного происхождения могут разделять данные через такой канал;
	\item именованный канал остается в файловой системе для дальнейшего использования и после того, как весь ввод/вывод сделан.
\end{itemize}


Существует два способа создания именованного канала:

Создать обычный файл, директорию или файл специального назначения с помощью системного вызова mknode (2), указав 0 в dev_t.

\begin{CCode}{mknode (2)}
	int mknod(
		const char *path, 
		mode_t mode, 
		dev_t dev
	); \end{CCode}

или воспользоваться функцией mkfifo (3)

\begin{CCode}{mkfifo (3)}
	int mkfifo(
		const char *pathname, 
		mode_t mode
	); \end{CCode}
	

	\chapter{Процессы}
	Именованные каналы во многом работают так же, как и неименованные каналы, но все же имеют несколько заметных отличий.

\begin{itemize}
	\item именованные каналы существуют в виде специального файла устройства в файловой системе;
	\item процессы различного происхождения могут разделять данные через такой канал;
	\item именованный канал остается в файловой системе для дальнейшего использования и после того, как весь ввод/вывод сделан.
\end{itemize}


Существует два способа создания именованного канала:

Создать обычный файл, директорию или файл специального назначения с помощью системного вызова mknode (2), указав 0 в dev_t.

\begin{CCode}{mknode (2)}
	int mknod(
		const char *path, 
		mode_t mode, 
		dev_t dev
	); \end{CCode}

или воспользоваться функцией mkfifo (3)

\begin{CCode}{mkfifo (3)}
	int mkfifo(
		const char *pathname, 
		mode_t mode
	); \end{CCode}

	
		\section{Атрибуты процесса}
		Именованные каналы во многом работают так же, как и неименованные каналы, но все же имеют несколько заметных отличий.

\begin{itemize}
	\item именованные каналы существуют в виде специального файла устройства в файловой системе;
	\item процессы различного происхождения могут разделять данные через такой канал;
	\item именованный канал остается в файловой системе для дальнейшего использования и после того, как весь ввод/вывод сделан.
\end{itemize}


Существует два способа создания именованного канала:

Создать обычный файл, директорию или файл специального назначения с помощью системного вызова mknode (2), указав 0 в dev_t.

\begin{CCode}{mknode (2)}
	int mknod(
		const char *path, 
		mode_t mode, 
		dev_t dev
	); \end{CCode}

или воспользоваться функцией mkfifo (3)

\begin{CCode}{mkfifo (3)}
	int mkfifo(
		const char *pathname, 
		mode_t mode
	); \end{CCode}


		\section{Жизненный цикл процесса}
		Именованные каналы во многом работают так же, как и неименованные каналы, но все же имеют несколько заметных отличий.

\begin{itemize}
	\item именованные каналы существуют в виде специального файла устройства в файловой системе;
	\item процессы различного происхождения могут разделять данные через такой канал;
	\item именованный канал остается в файловой системе для дальнейшего использования и после того, как весь ввод/вывод сделан.
\end{itemize}


Существует два способа создания именованного канала:

Создать обычный файл, директорию или файл специального назначения с помощью системного вызова mknode (2), указав 0 в dev_t.

\begin{CCode}{mknode (2)}
	int mknod(
		const char *path, 
		mode_t mode, 
		dev_t dev
	); \end{CCode}

или воспользоваться функцией mkfifo (3)

\begin{CCode}{mkfifo (3)}
	int mkfifo(
		const char *pathname, 
		mode_t mode
	); \end{CCode}


		\section{Создание процессов}
		Именованные каналы во многом работают так же, как и неименованные каналы, но все же имеют несколько заметных отличий.

\begin{itemize}
	\item именованные каналы существуют в виде специального файла устройства в файловой системе;
	\item процессы различного происхождения могут разделять данные через такой канал;
	\item именованный канал остается в файловой системе для дальнейшего использования и после того, как весь ввод/вывод сделан.
\end{itemize}


Существует два способа создания именованного канала:

Создать обычный файл, директорию или файл специального назначения с помощью системного вызова mknode (2), указав 0 в dev_t.

\begin{CCode}{mknode (2)}
	int mknod(
		const char *path, 
		mode_t mode, 
		dev_t dev
	); \end{CCode}

или воспользоваться функцией mkfifo (3)

\begin{CCode}{mkfifo (3)}
	int mkfifo(
		const char *pathname, 
		mode_t mode
	); \end{CCode}


			\subsection{Семейство системных вызовов fork}
			Именованные каналы во многом работают так же, как и неименованные каналы, но все же имеют несколько заметных отличий.

\begin{itemize}
	\item именованные каналы существуют в виде специального файла устройства в файловой системе;
	\item процессы различного происхождения могут разделять данные через такой канал;
	\item именованный канал остается в файловой системе для дальнейшего использования и после того, как весь ввод/вывод сделан.
\end{itemize}


Существует два способа создания именованного канала:

Создать обычный файл, директорию или файл специального назначения с помощью системного вызова mknode (2), указав 0 в dev_t.

\begin{CCode}{mknode (2)}
	int mknod(
		const char *path, 
		mode_t mode, 
		dev_t dev
	); \end{CCode}

или воспользоваться функцией mkfifo (3)

\begin{CCode}{mkfifo (3)}
	int mkfifo(
		const char *pathname, 
		mode_t mode
	); \end{CCode}


			\subsection{Семейство системных вызовов exec}
			Именованные каналы во многом работают так же, как и неименованные каналы, но все же имеют несколько заметных отличий.

\begin{itemize}
	\item именованные каналы существуют в виде специального файла устройства в файловой системе;
	\item процессы различного происхождения могут разделять данные через такой канал;
	\item именованный канал остается в файловой системе для дальнейшего использования и после того, как весь ввод/вывод сделан.
\end{itemize}


Существует два способа создания именованного канала:

Создать обычный файл, директорию или файл специального назначения с помощью системного вызова mknode (2), указав 0 в dev_t.

\begin{CCode}{mknode (2)}
	int mknod(
		const char *path, 
		mode_t mode, 
		dev_t dev
	); \end{CCode}

или воспользоваться функцией mkfifo (3)

\begin{CCode}{mkfifo (3)}
	int mkfifo(
		const char *pathname, 
		mode_t mode
	); \end{CCode}


	\chapter{Межпроцессное взаимодействие}
	Именованные каналы во многом работают так же, как и неименованные каналы, но все же имеют несколько заметных отличий.

\begin{itemize}
	\item именованные каналы существуют в виде специального файла устройства в файловой системе;
	\item процессы различного происхождения могут разделять данные через такой канал;
	\item именованный канал остается в файловой системе для дальнейшего использования и после того, как весь ввод/вывод сделан.
\end{itemize}


Существует два способа создания именованного канала:

Создать обычный файл, директорию или файл специального назначения с помощью системного вызова mknode (2), указав 0 в dev_t.

\begin{CCode}{mknode (2)}
	int mknod(
		const char *path, 
		mode_t mode, 
		dev_t dev
	); \end{CCode}

или воспользоваться функцией mkfifo (3)

\begin{CCode}{mkfifo (3)}
	int mkfifo(
		const char *pathname, 
		mode_t mode
	); \end{CCode}
	
	
		\section{Семафоры}
		Именованные каналы во многом работают так же, как и неименованные каналы, но все же имеют несколько заметных отличий.

\begin{itemize}
	\item именованные каналы существуют в виде специального файла устройства в файловой системе;
	\item процессы различного происхождения могут разделять данные через такой канал;
	\item именованный канал остается в файловой системе для дальнейшего использования и после того, как весь ввод/вывод сделан.
\end{itemize}


Существует два способа создания именованного канала:

Создать обычный файл, директорию или файл специального назначения с помощью системного вызова mknode (2), указав 0 в dev_t.

\begin{CCode}{mknode (2)}
	int mknod(
		const char *path, 
		mode_t mode, 
		dev_t dev
	); \end{CCode}

или воспользоваться функцией mkfifo (3)

\begin{CCode}{mkfifo (3)}
	int mkfifo(
		const char *pathname, 
		mode_t mode
	); \end{CCode}


		\section{Каналы}
		Именованные каналы во многом работают так же, как и неименованные каналы, но все же имеют несколько заметных отличий.

\begin{itemize}
	\item именованные каналы существуют в виде специального файла устройства в файловой системе;
	\item процессы различного происхождения могут разделять данные через такой канал;
	\item именованный канал остается в файловой системе для дальнейшего использования и после того, как весь ввод/вывод сделан.
\end{itemize}


Существует два способа создания именованного канала:

Создать обычный файл, директорию или файл специального назначения с помощью системного вызова mknode (2), указав 0 в dev_t.

\begin{CCode}{mknode (2)}
	int mknod(
		const char *path, 
		mode_t mode, 
		dev_t dev
	); \end{CCode}

или воспользоваться функцией mkfifo (3)

\begin{CCode}{mkfifo (3)}
	int mkfifo(
		const char *pathname, 
		mode_t mode
	); \end{CCode}
	
		
			\subsection{Неименованные каналы}
			Именованные каналы во многом работают так же, как и неименованные каналы, но все же имеют несколько заметных отличий.

\begin{itemize}
	\item именованные каналы существуют в виде специального файла устройства в файловой системе;
	\item процессы различного происхождения могут разделять данные через такой канал;
	\item именованный канал остается в файловой системе для дальнейшего использования и после того, как весь ввод/вывод сделан.
\end{itemize}


Существует два способа создания именованного канала:

Создать обычный файл, директорию или файл специального назначения с помощью системного вызова mknode (2), указав 0 в dev_t.

\begin{CCode}{mknode (2)}
	int mknod(
		const char *path, 
		mode_t mode, 
		dev_t dev
	); \end{CCode}

или воспользоваться функцией mkfifo (3)

\begin{CCode}{mkfifo (3)}
	int mkfifo(
		const char *pathname, 
		mode_t mode
	); \end{CCode}

			
			\subsection{Именованные каналы}
			Именованные каналы во многом работают так же, как и неименованные каналы, но все же имеют несколько заметных отличий.

\begin{itemize}
	\item именованные каналы существуют в виде специального файла устройства в файловой системе;
	\item процессы различного происхождения могут разделять данные через такой канал;
	\item именованный канал остается в файловой системе для дальнейшего использования и после того, как весь ввод/вывод сделан.
\end{itemize}


Существует два способа создания именованного канала:

Создать обычный файл, директорию или файл специального назначения с помощью системного вызова mknode (2), указав 0 в dev_t.

\begin{CCode}{mknode (2)}
	int mknod(
		const char *path, 
		mode_t mode, 
		dev_t dev
	); \end{CCode}

или воспользоваться функцией mkfifo (3)

\begin{CCode}{mkfifo (3)}
	int mkfifo(
		const char *pathname, 
		mode_t mode
	); \end{CCode}
	

		\section{Сокеты}
		Именованные каналы во многом работают так же, как и неименованные каналы, но все же имеют несколько заметных отличий.

\begin{itemize}
	\item именованные каналы существуют в виде специального файла устройства в файловой системе;
	\item процессы различного происхождения могут разделять данные через такой канал;
	\item именованный канал остается в файловой системе для дальнейшего использования и после того, как весь ввод/вывод сделан.
\end{itemize}


Существует два способа создания именованного канала:

Создать обычный файл, директорию или файл специального назначения с помощью системного вызова mknode (2), указав 0 в dev_t.

\begin{CCode}{mknode (2)}
	int mknod(
		const char *path, 
		mode_t mode, 
		dev_t dev
	); \end{CCode}

или воспользоваться функцией mkfifo (3)

\begin{CCode}{mkfifo (3)}
	int mkfifo(
		const char *pathname, 
		mode_t mode
	); \end{CCode}

		
		\section{Сигналы}
		Именованные каналы во многом работают так же, как и неименованные каналы, но все же имеют несколько заметных отличий.

\begin{itemize}
	\item именованные каналы существуют в виде специального файла устройства в файловой системе;
	\item процессы различного происхождения могут разделять данные через такой канал;
	\item именованный канал остается в файловой системе для дальнейшего использования и после того, как весь ввод/вывод сделан.
\end{itemize}


Существует два способа создания именованного канала:

Создать обычный файл, директорию или файл специального назначения с помощью системного вызова mknode (2), указав 0 в dev_t.

\begin{CCode}{mknode (2)}
	int mknod(
		const char *path, 
		mode_t mode, 
		dev_t dev
	); \end{CCode}

или воспользоваться функцией mkfifo (3)

\begin{CCode}{mkfifo (3)}
	int mkfifo(
		const char *pathname, 
		mode_t mode
	); \end{CCode}

		
			\subsection{Обработка сигналов}
			Именованные каналы во многом работают так же, как и неименованные каналы, но все же имеют несколько заметных отличий.

\begin{itemize}
	\item именованные каналы существуют в виде специального файла устройства в файловой системе;
	\item процессы различного происхождения могут разделять данные через такой канал;
	\item именованный канал остается в файловой системе для дальнейшего использования и после того, как весь ввод/вывод сделан.
\end{itemize}


Существует два способа создания именованного канала:

Создать обычный файл, директорию или файл специального назначения с помощью системного вызова mknode (2), указав 0 в dev_t.

\begin{CCode}{mknode (2)}
	int mknod(
		const char *path, 
		mode_t mode, 
		dev_t dev
	); \end{CCode}

или воспользоваться функцией mkfifo (3)

\begin{CCode}{mkfifo (3)}
	int mkfifo(
		const char *pathname, 
		mode_t mode
	); \end{CCode}
	
		
		\section{Очереди сообщений}
		Именованные каналы во многом работают так же, как и неименованные каналы, но все же имеют несколько заметных отличий.

\begin{itemize}
	\item именованные каналы существуют в виде специального файла устройства в файловой системе;
	\item процессы различного происхождения могут разделять данные через такой канал;
	\item именованный канал остается в файловой системе для дальнейшего использования и после того, как весь ввод/вывод сделан.
\end{itemize}


Существует два способа создания именованного канала:

Создать обычный файл, директорию или файл специального назначения с помощью системного вызова mknode (2), указав 0 в dev_t.

\begin{CCode}{mknode (2)}
	int mknod(
		const char *path, 
		mode_t mode, 
		dev_t dev
	); \end{CCode}

или воспользоваться функцией mkfifo (3)

\begin{CCode}{mkfifo (3)}
	int mkfifo(
		const char *pathname, 
		mode_t mode
	); \end{CCode}
	

		\section{Разделяемая память}
		Именованные каналы во многом работают так же, как и неименованные каналы, но все же имеют несколько заметных отличий.

\begin{itemize}
	\item именованные каналы существуют в виде специального файла устройства в файловой системе;
	\item процессы различного происхождения могут разделять данные через такой канал;
	\item именованный канал остается в файловой системе для дальнейшего использования и после того, как весь ввод/вывод сделан.
\end{itemize}


Существует два способа создания именованного канала:

Создать обычный файл, директорию или файл специального назначения с помощью системного вызова mknode (2), указав 0 в dev_t.

\begin{CCode}{mknode (2)}
	int mknod(
		const char *path, 
		mode_t mode, 
		dev_t dev
	); \end{CCode}

или воспользоваться функцией mkfifo (3)

\begin{CCode}{mkfifo (3)}
	int mkfifo(
		const char *pathname, 
		mode_t mode
	); \end{CCode}
	

		\section{Механизм STREAMS}
		Именованные каналы во многом работают так же, как и неименованные каналы, но все же имеют несколько заметных отличий.

\begin{itemize}
	\item именованные каналы существуют в виде специального файла устройства в файловой системе;
	\item процессы различного происхождения могут разделять данные через такой канал;
	\item именованный канал остается в файловой системе для дальнейшего использования и после того, как весь ввод/вывод сделан.
\end{itemize}


Существует два способа создания именованного канала:

Создать обычный файл, директорию или файл специального назначения с помощью системного вызова mknode (2), указав 0 в dev_t.

\begin{CCode}{mknode (2)}
	int mknod(
		const char *path, 
		mode_t mode, 
		dev_t dev
	); \end{CCode}

или воспользоваться функцией mkfifo (3)

\begin{CCode}{mkfifo (3)}
	int mkfifo(
		const char *pathname, 
		mode_t mode
	); \end{CCode}
	

	\chapter{Потоки}
	Именованные каналы во многом работают так же, как и неименованные каналы, но все же имеют несколько заметных отличий.

\begin{itemize}
	\item именованные каналы существуют в виде специального файла устройства в файловой системе;
	\item процессы различного происхождения могут разделять данные через такой канал;
	\item именованный канал остается в файловой системе для дальнейшего использования и после того, как весь ввод/вывод сделан.
\end{itemize}


Существует два способа создания именованного канала:

Создать обычный файл, директорию или файл специального назначения с помощью системного вызова mknode (2), указав 0 в dev_t.

\begin{CCode}{mknode (2)}
	int mknod(
		const char *path, 
		mode_t mode, 
		dev_t dev
	); \end{CCode}

или воспользоваться функцией mkfifo (3)

\begin{CCode}{mkfifo (3)}
	int mkfifo(
		const char *pathname, 
		mode_t mode
	); \end{CCode}


		\section{Легковесные процессы}
		Именованные каналы во многом работают так же, как и неименованные каналы, но все же имеют несколько заметных отличий.

\begin{itemize}
	\item именованные каналы существуют в виде специального файла устройства в файловой системе;
	\item процессы различного происхождения могут разделять данные через такой канал;
	\item именованный канал остается в файловой системе для дальнейшего использования и после того, как весь ввод/вывод сделан.
\end{itemize}


Существует два способа создания именованного канала:

Создать обычный файл, директорию или файл специального назначения с помощью системного вызова mknode (2), указав 0 в dev_t.

\begin{CCode}{mknode (2)}
	int mknod(
		const char *path, 
		mode_t mode, 
		dev_t dev
	); \end{CCode}

или воспользоваться функцией mkfifo (3)

\begin{CCode}{mkfifo (3)}
	int mkfifo(
		const char *pathname, 
		mode_t mode
	); \end{CCode}


		\section{Межпоточное взаимодействие}
		Именованные каналы во многом работают так же, как и неименованные каналы, но все же имеют несколько заметных отличий.

\begin{itemize}
	\item именованные каналы существуют в виде специального файла устройства в файловой системе;
	\item процессы различного происхождения могут разделять данные через такой канал;
	\item именованный канал остается в файловой системе для дальнейшего использования и после того, как весь ввод/вывод сделан.
\end{itemize}


Существует два способа создания именованного канала:

Создать обычный файл, директорию или файл специального назначения с помощью системного вызова mknode (2), указав 0 в dev_t.

\begin{CCode}{mknode (2)}
	int mknod(
		const char *path, 
		mode_t mode, 
		dev_t dev
	); \end{CCode}

или воспользоваться функцией mkfifo (3)

\begin{CCode}{mkfifo (3)}
	int mkfifo(
		const char *pathname, 
		mode_t mode
	); \end{CCode}
	
		
			\subsection{Общее адресное пространство}
			Именованные каналы во многом работают так же, как и неименованные каналы, но все же имеют несколько заметных отличий.

\begin{itemize}
	\item именованные каналы существуют в виде специального файла устройства в файловой системе;
	\item процессы различного происхождения могут разделять данные через такой канал;
	\item именованный канал остается в файловой системе для дальнейшего использования и после того, как весь ввод/вывод сделан.
\end{itemize}


Существует два способа создания именованного канала:

Создать обычный файл, директорию или файл специального назначения с помощью системного вызова mknode (2), указав 0 в dev_t.

\begin{CCode}{mknode (2)}
	int mknod(
		const char *path, 
		mode_t mode, 
		dev_t dev
	); \end{CCode}

или воспользоваться функцией mkfifo (3)

\begin{CCode}{mkfifo (3)}
	int mkfifo(
		const char *pathname, 
		mode_t mode
	); \end{CCode}
	
		
			\subsection{Переменные volatile}
			Именованные каналы во многом работают так же, как и неименованные каналы, но все же имеют несколько заметных отличий.

\begin{itemize}
	\item именованные каналы существуют в виде специального файла устройства в файловой системе;
	\item процессы различного происхождения могут разделять данные через такой канал;
	\item именованный канал остается в файловой системе для дальнейшего использования и после того, как весь ввод/вывод сделан.
\end{itemize}


Существует два способа создания именованного канала:

Создать обычный файл, директорию или файл специального назначения с помощью системного вызова mknode (2), указав 0 в dev_t.

\begin{CCode}{mknode (2)}
	int mknod(
		const char *path, 
		mode_t mode, 
		dev_t dev
	); \end{CCode}

или воспользоваться функцией mkfifo (3)

\begin{CCode}{mkfifo (3)}
	int mkfifo(
		const char *pathname, 
		mode_t mode
	); \end{CCode}
		
	
			\subsection{Мьютексы}
			Именованные каналы во многом работают так же, как и неименованные каналы, но все же имеют несколько заметных отличий.

\begin{itemize}
	\item именованные каналы существуют в виде специального файла устройства в файловой системе;
	\item процессы различного происхождения могут разделять данные через такой канал;
	\item именованный канал остается в файловой системе для дальнейшего использования и после того, как весь ввод/вывод сделан.
\end{itemize}


Существует два способа создания именованного канала:

Создать обычный файл, директорию или файл специального назначения с помощью системного вызова mknode (2), указав 0 в dev_t.

\begin{CCode}{mknode (2)}
	int mknod(
		const char *path, 
		mode_t mode, 
		dev_t dev
	); \end{CCode}

или воспользоваться функцией mkfifo (3)

\begin{CCode}{mkfifo (3)}
	int mkfifo(
		const char *pathname, 
		mode_t mode
	); \end{CCode}
	
			
			\subsection{rwlock}
			Именованные каналы во многом работают так же, как и неименованные каналы, но все же имеют несколько заметных отличий.

\begin{itemize}
	\item именованные каналы существуют в виде специального файла устройства в файловой системе;
	\item процессы различного происхождения могут разделять данные через такой канал;
	\item именованный канал остается в файловой системе для дальнейшего использования и после того, как весь ввод/вывод сделан.
\end{itemize}


Существует два способа создания именованного канала:

Создать обычный файл, директорию или файл специального назначения с помощью системного вызова mknode (2), указав 0 в dev_t.

\begin{CCode}{mknode (2)}
	int mknod(
		const char *path, 
		mode_t mode, 
		dev_t dev
	); \end{CCode}

или воспользоваться функцией mkfifo (3)

\begin{CCode}{mkfifo (3)}
	int mkfifo(
		const char *pathname, 
		mode_t mode
	); \end{CCode}
	
		
		
		
\addcontentsline{toc}{chapter}{Литература}
\newpage

\begin{thebibliography}{99}

\bibitem{OSUNIX} Робачевский А. М., Немнюгин С. А., Стесик О. Л. Операционная система UNIX. — 2-еизд., перераб. и доп.— СПб.: БХВ-Петербург, 2008.

\bibitem{INSIDEUNIX} Вахалия Ю. UNIX изнутри — СПб.: Питер, 2003.
\end{thebibliography}
	
\end{document}