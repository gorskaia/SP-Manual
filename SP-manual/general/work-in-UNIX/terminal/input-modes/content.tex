Терминал умеет работать только в двух режимах --- каноническом и неканоническом. Различаются они тем, что в неканоническом режиме введенный пользователем текст сразу же посимвольно отправляется на обработку операционной системе, а в каноническом режиме введенный пользователем текст сохраняется в так называемом line-буфере --- буфере терминала --- и отправляется в ОС только после нажатия клавиши enter.

\textit{Если проводить аналогию с написанием терминала для базовой ЭВМ, то в каноническом режиме был бы реализован буфер для хранения строки перед отправкой ее в БЭВМ. Раньше смысл этого заключался в том, чтобы сократить количество прерываний, потому что можно было вызвать одно прерывание “строка готова”, и процессор начинал эту строчку читать. С ускорением вычислительных ресурсов, увеличением их количества и введением таких вещей, как DMA (Direct Memory Access --- прямой доступ к памяти), проблема себя исчерпала, и поэтому отпала необходимость в line-буфере.
}

Возвращаясь к каноническому и неканоническому режиму, помним, что мы либо вводим построчно, либо посимвольно. Разницу в этом мы прочувствуем несколько далее.