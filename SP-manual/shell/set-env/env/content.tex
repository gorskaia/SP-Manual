\begin{defi}{env}
	отдельный файл-утилита, находится в user/bin. Происходит от слова environment - окружение. Утилита env позволяет запускать команды не с текущим окружением, а с любым созданным вами.
\end{defi}

У env есть всего один ключ -i (от слова initial - начальный). Этот ключ значит, что все окружение запускаемой команды будет перечислено в аргументах env. Других переменных в окружении у этой команды не будет.

Тонкий момент с утилитой env и ее использовании. Очень часто ее используют в shebang. Например, если вы не знаете где в системе пользователя лежит какой-то язык программирования и знаете, что есть каталог \user\bin\env, то вы можете написать в shebang следующую конструкцию: #!\usr\bin\env python. Это позволит нам запустить python не зная где он расположен на системе. Фактически написать скрипт, который будет легко переносим.