Как вы можете помнить из других курсов, компиляция программы нужна для получения исполняемого файла. Сам процесс компиляции состоит из этапов препроцессинга, трансляции и компоновки.

На этапе препроцессинга осуществляется подстановка исходного тела макросов на место их имен. Далее происходит трансляция полученного кода в язык более низкого уровня. Последний этап	 -- компоновка, устанавливает связи между отдельными различными файлами и генерирует из них один - исполняемый.