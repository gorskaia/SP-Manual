Как мы уже говорили, для того чтобы работать с файлом, нужно поместить информацию об этом файле в таблицу файловых дескрипторов. Для этой цели в ОС UNIX существует системный вызов \textbf{open (2)}. 

\begin{CCode}{open (2)}
	int open(const char *path,		/* path to file */ 
		int oflag,					/* access mode */
		/* mode_t mode */			/* used only on file creation */
	); \end{CCode}

В качестве аргументов этот системный вызов принимает путь к файлу, режим доступа и права доступа. Права доступа используются в случае, если мы вызываем open (2) с целью создания файла (в некоторых реализациях библиотеки Си, альтернативой является \textbf{creat (2)}). 
Системный вызов open (2) возвращает номер файлового дескриптора или код ошибки.

\textbf{Приведем основные режимы доступа}

\begin{center}
	\begin{tabular}{c|l}
		\textbf{Значение oflag} & \textbf{Назначение} \\
		\hline
		O\_RDONLY	&	только для чтения \\
		\hline
		O\_WRONLY	&	только для записи \\
		\hline
		O\_RDWR		&	для записи/чтения \\
		\hline
		O\_CREAT	&	создать, если не существует \\
		\hline
		O\_APPEND	&	запись с конца \\
		\hline 
		O\_TRUNC	&	запись с начала (фактически обрезать файл до нулевой длины)
	\end{tabular}
\end{center}

\textbf{Какие могут произойти ошибки при открытии файла?}

При открытии файла могут произойти ошибки в случае, если мы не имеем доступа к файлу или не нашлось устройство, на котором этот файл располагается. Также ошибка может произойти, если в таблице файловых дескрипторов больше нет свободных строк.

Когда файл станет нам окончательно не нужен, можно закрыть его с помощью системного вызова close(2). Тем самым мы освободим одну строку в таблице файловых дескрипторов. Как вы помните, эта таблица имеет ограничения, поэтому это может быть важным. Возвращает 0 или код ошибки.

\begin{CCode}{close(2)}
	int close( 
		int fildes /* descriptor of file */ 
	); \end{CCode}
	
\textbf{В каком случае close может вернуть код ошибки?}

Единственный вариант --- при передаче неверного файлового дескриптора. Т.е. этот файловый дескриптор, который не был открыт в таблице.
